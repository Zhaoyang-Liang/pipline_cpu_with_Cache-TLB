\documentclass[a4paper]{article}
\input{style/ch_xelatex.tex}
% "define" Scala
\usepackage[T1]{fontenc}  
\usepackage[scaled=0.82]{beramono}  
\usepackage{microtype} 

\sbox0{\small\ttfamily A}
\edef\mybasewidth{\the\wd0 }

\lstdefinelanguage{scala}{
  morekeywords={abstract,case,catch,class,def,%
    do,else,extends,false,final,finally,%
    for,if,implicit,import,match,mixin,%
    new,null,object,override,package,%
    private,protected,requires,return,sealed,%
    super,this,throw,trait,true,try,%
    type,val,var,while,with,yield},
  sensitive=true,
  morecomment=[l]{//},
  morecomment=[n]{/*}{*/},
  morestring=[b]",
  morestring=[b]',
  morestring=[b]"""
}

\usepackage{color}
\definecolor{dkgreen}{rgb}{0,0.6,0}
\definecolor{gray}{rgb}{0.5,0.5,0.5}
\definecolor{mauve}{rgb}{0.58,0,0.82}

% Default settings for code listings
\lstset{frame=tb,
  language=scala,
  aboveskip=3mm,
  belowskip=3mm,
  showstringspaces=false,
  columns=fixed, % basewidth=\mybasewidth,
  basicstyle={\small\ttfamily},
  numbers=none,
  numberstyle=\footnotesize\color{gray},
  % identifierstyle=\color{red},
  keywordstyle=\color{blue},
  commentstyle=\color{dkgreen},
  stringstyle=\color{mauve},
  frame=single,
  breaklines=true,
  breakatwhitespace=true,
  procnamekeys={def, val, var, class, trait, object, extends},
  procnamestyle=\ttfamily\color{red},
  tabsize=2
}

\lstnewenvironment{scala}[1][]
{\lstset{language=scala,#1}}
{}
\lstnewenvironment{cpp}[1][]
{\lstset{language=C++,#1}}
{}
\lstnewenvironment{bash}[1][]
{\lstset{language=bash,#1}}
{}
\lstnewenvironment{verilog}[1][]
{\lstset{language=verilog,#1}}
{}



\graphicspath{ {images/} }
\usepackage{ctex}
\usepackage{graphicx}
\usepackage{color,framed}%文本框
\usepackage{listings}
\usepackage{caption}

\usepackage{hyperref}
\hypersetup{hidelinks,
	colorlinks=true,
	allcolors=black,
	pdfstartview=Fit,
	breaklinks=true}


\lstdefinestyle{mystyle}{
  keywordstyle=\color{orange},  % 设置关键词颜色为橙色
 morekeywords={set, nmap, msfconsole,exploit},           % 仅将 set 设置为橙色
}

\usepackage{amssymb}
\usepackage{enumerate}
\usepackage{xcolor}
\usepackage{bm} 
\usepackage{lastpage}%获得总页数
\usepackage{fancyhdr}
\usepackage{tabularx}  
\usepackage{geometry}
\usepackage{minted}
\usepackage{graphics}
\usepackage{subfigure}
\usepackage{float}
\usepackage{pdfpages}
\usepackage{pgfplots}
\pgfplotsset{width=10cm,compat=1.9}
\usepackage{multirow}
\usepackage{footnote}
\usepackage{booktabs}
\usepackage{listings}

%-----------------------伪代码------------------
\usepackage{algorithm}  
\usepackage{algorithmicx}  
\usepackage{algpseudocode}  
\floatname{algorithm}{Algorithm}  
\renewcommand{\algorithmicrequire}{\textbf{Input:}}  
\renewcommand{\algorithmicensure}{\textbf{Output:}} 
\usepackage{lipsum}  
\makeatletter
\newenvironment{breakablealgorithm}
  {% \begin{breakablealgorithm}
  \begin{center}
     \refstepcounter{algorithm}% New algorithm
     \hrule height.8pt depth0pt \kern2pt% \@fs@pre for \@fs@ruled
     \renewcommand{\caption}[2][\relax]{% Make a new \caption
      {\raggedright\textbf{\ALG@name~\thealgorithm} ##2\par}%
      \ifx\relax##1\relax % #1 is \relax
         \addcontentsline{loa}{algorithm}{\protect\numberline{\thealgorithm}##2}%
      \else % #1 is not \relax
         \addcontentsline{loa}{algorithm}{\protect\numberline{\thealgorithm}##1}%
      \fi
      \kern2pt\hrule\kern2pt
     }
  }{% \end{breakablealgorithm}
     \kern2pt\hrule\relax% \@fs@post for \@fs@ruled
  \end{center}
  }
\makeatother
%------------------------代码-------------------
\RequirePackage{listings}
\RequirePackage{xcolor}
\definecolor{dkgreen}{rgb}{0,0.6,0}
\definecolor{gray}{rgb}{0.5,0.5,0.5}
\definecolor{mauve}{rgb}{0.58,0,0.82}
\lstset{
	numbers=left,  
	frame=tb,
	aboveskip=3mm,
	belowskip=3mm,
	showstringspaces=false,
	columns=flexible,
	framerule=1pt,
	rulecolor=\color{gray!35},
	backgroundcolor=\color{gray!5},
	basicstyle={\ttfamily},
	numberstyle=\tiny\color{gray},
	keywordstyle=\color{blue},
	commentstyle=\color{dkgreen},
	stringstyle=\color{mauve},
	breaklines=true,
	breakatwhitespace=true,
	tabsize=3,
}


%-------------------------页面边距--------------
\geometry{a4paper,left=2.3cm,right=2.3cm,top=2.7cm,bottom=2.7cm}
%-------------------------页眉页脚--------------
\usepackage{fancyhdr}
\pagestyle{fancy}
\lhead{\kaishu }
\chead{}
\rhead{{\CJKfontspec{simkai.ttf} }}
\lfoot{}
\cfoot{\thepage}
\rfoot{}
\renewcommand{\headrulewidth}{0pt}  
\renewcommand{\footrulewidth}{0pt}
\newcommand{\HRule}{\rule{\linewidth}{0.5mm}}
\newcommand{\HRulegrossa}{\rule{\linewidth}{1.2mm}}
\setlength{\textfloatsep}{10mm}
%--------------------文档内容--------------------
\renewcommand{\contentsname}{目\ 录}
\renewcommand{\appendixname}{附录}
\renewcommand{\appendixpagename}{附录}
\renewcommand{\refname}{参考文献} 
\renewcommand{\figurename}{图}
\renewcommand{\tablename}{表}
\renewcommand{\today}{\number\year 年 \number\month 月 \number\day 日}

\renewcommand {\thefigure}{\thesection{}.\arabic{figure}}%图片按章标号
\renewcommand{\figurename}{图}
\renewcommand{\contentsname}{目录}  
\cfoot{\thepage\ of \pageref{LastPage}}%当前页 of 总页数

\renewcommand{\abstractname}{\textbf{\Large 摘要}} % 调整摘要标题的字体大小
\renewcommand {\thefigure}{\thesection{}.\arabic{figure}}%图片按章标号

\usepackage{xcolor}
% 代码样式设置(Verilog)

\lstdefinelanguage{Verilog}{
	keywords=[1]{module, endmodule, input, output,  reg, always, begin, end, if, else, case, endcase, assign},
	keywordstyle=[1]\color{blue}\bfseries,
	keywords=[2]{and, or, nand, nor, xor, xnor, buf,wire, not},
	keywordstyle=[2]\color{red},
	sensitive=false, % Verilog 大小写不敏感
	morecomment=[l][commentstyle]{//},
	morecomment=[s]{/*}{*/}, % 块注释
	commentstyle=\color{green}, % 注释风格
	%morestring=[b]", % 字符串
	morecomment=[l][commentstyle]{'},
	morecomment=[l][commentstyle]{"},
	%morestring=[b]' % 字符
}
\definecolor{darkgreen}{rgb}{0,0.5,0}
\lstset{
	language=Verilog,
	basicstyle=\ttfamily\small, % 代码的基本样式
	commentstyle=\color{darkgreen}, % 注释样式
	stringstyle=\color{orange}, % 字符串样式
	numberstyle=\color{gray}, % 设置数字颜色
	numbers=left, % 行号在左侧
	numberstyle=\tiny\color{gray}, % 行号样式
	stepnumber=1, % 行号步进
	breaklines=true, % 自动换行
	frame=single, % 代码框
	tabsize=4, % Tab 大小
	showstringspaces=false, % 不显示字符串中的空格
	extendedchars=true, % 支持扩展字符
}

\usepackage{booktabs}
\usepackage{geometry}
\geometry{left=2.5cm,right=2.5cm,top=3cm,bottom=3cm}

\begin{document}

\renewcommand{\figurename}{图}
\renewcommand{\contentsname}{目录}  
\cfoot{\thepage\ of \pageref{LastPage}}


\renewcommand{\abstractname}{\textbf{\Large 摘要}} 

\begin{center}
    \huge{\textbf{在五级流水CPU上实现AXI总线}}
\end{center}

\begin{center}
    \textbf{姓名}:\underline{梁朝阳} \quad
    \textbf{专业}:\underline{密码科学与技术}
\end{center}

\tableofcontents

\vspace*{1cm}

\newpage


\noindent{\Large\textbf{摘要}}

\vspace{1em}

因为我想接着在五级流水上实现这个AXI,所以主要自己找了一些资料。

\textbf{参考资料:} 我主要参考了b站上的一个视频,\textbf{\href{https://www.bilibili.com/video/BV1mD4y1p7UK?vd_source=37de53df519eb1b7b2ec347aee6a998f}{【如何科学的设计FPGA:实现AXI总线自由之AXI解读】}}。首先照着这个手敲了一遍AXI,然后想办法怎么把AXI接到CPU上。单独的AXI不是很难,但是接入到五级流水的时候改动非常大。


\textbf{AXI总线实现:} 首先系统梳理了 AXI 协议的五通道握手机制、突发传输模式及其在 CPU 访存中的应用场景;随后分别实现了 AXI4-Full Master 与 AXI RAM Slave,支持 INCR、FIXED 与 WRAP 突发方式、字节写使能以及读写并行操作。在与五级流水 CPU 集成过程中,对取指阶段(IF)与访存阶段(MEM)进行了全面改造,通过状态机、握手保持、忙/完成标志及流水线 over/allow 机制,实现了 CPU 与 AXI 总线的无缝交互。

\textbf{CPU加上总线:} 实验过程中解决了 PC 无法更新、指令无法进入流水线、访存阶段死锁、Slave 超读等多个问题,最后实现了 AXI–CPU 通路。最后CPU 能够正确完成指令取指与数据访存,流水线各级协同工作良好,功能仿真结果与期望一致。

\vspace{1em}
\noindent\textbf{关键词:} AXI 总线;五级流水 CPU
\newpage
\section{实验要求}
参考《CPU设计实战》的第8章,自己动手实现总线接口相关功能。要求:
\begin{enumerate}
    \item 根据自己情况,类SRAM总线和AXI总线完成一项支持即可,建议选择AXI总线(需要查阅AXI总线文档)。
    \item 虽然说不建议使用当前五级流水的代码了,但是我学习后想保持一个实验的连续性,而且我想把这个CPU实现完全,因为还是采用了这个CPU的代码。
\end{enumerate}

\section{AXI原理}
\subsection{总览}

AXI 总线是一种突发总线,突发传输。一直连续的传输,比如说突发8次传输,就是指传输数据连续的传输。(突发传输的地址是连续的,数据是连续的)

\begin{enumerate}
    \item 读:地址控制(主机(Master)发送读地址)-> 从机(Slave)发送数据。
    \item 写:地址控制(主机发送读地址)-> 主机发送数据 -> 接受完后response。
\end{enumerate}

\section{AXI的五个通道的信号}

\subsection{全局信号}
\begin{table}[H]
    \centering
    \caption{全局信号 | Global Signals}
    \label{tab:global_signals}
    \begin{tabular}{cccc}
        \toprule
        \textbf{信号} & \textbf{来源} & \textbf{描述} \\
        \midrule
        ACLK & 时钟源 & 全局时钟信号,所有信号在时钟上升沿采样 \\
        ARESETn & 复位源 & 全局复位信号,低电平有效\\
        \bottomrule
    \end{tabular}
\end{table}

\subsection{写地址}
\begin{table}[H]
    \centering
    \caption{写地址通道信号|Write address channel signals}
    \label{tab:write_address_channel_signals}
    \begin{tabular}{cccc}
        \toprule
        \textbf{通道}&  \textbf{来源} & \textbf{宽度} & \textbf{描述} \\
        \midrule
        AWID[3:0] & Master & 4 & 写地址ID \\
        \textbf{\textcolor{red}{AWADDR[31:0]}} & Master & 32 & 写地址 \\
        \textbf{\textcolor{red}{AWLEN[3:0]}} & Master & 4 & 给出了突发传输几次 \\
        \textbf{\textcolor{red}{AWSIZE[2:0]}} & Master & 3 & 突发传输的数据宽度 。例如有32个字节,那么size是4 。\\
        AWBURST[1:0] & Master & 2 & 突发传输类型 $\dagger $\\
        AWLOCK[1:0] & -- & 2 & AXI不支持锁 \\
        AWCACHE[3:0] & Master & 4 & cache类型 \\
        AWPROT[2:0] & Master & 3 & Protection type. (不是端口,而是保护) \\
        AWVALID & Master & 1 & 写地址有效 \\    
        AWREADY & Slave & 1 & 写地址准备好 $\ddagger$ \\
        \bottomrule
    \end{tabular}
\end{table}
$\dagger$ 固定自增扫描打包,00: fixed, 01: increment, 10: wrapping. 

$\ddagger$ AWVALID \& AWREADY为1 时可以数据有效

\subsection{写数据}

\begin{table}[H]
    \centering
    \caption{写地址通道信号|Write address channel signals}
    \label{tab:write_address_channel_signals}
    \begin{tabular}{cccc}
        \toprule
        \textbf{通道}&  \textbf{来源} & \textbf{宽度} & \textbf{描述} \\
        \midrule
        WID[3:0] & Master & 4 & 写数据ID \\
        \textbf{\textcolor{red}{WDATA[31:0]}} & Master & 32 & 写数据 \\
        \textbf{\textcolor{red}{WSTRB[3:0]}} & Master & 4 & (掩码/闪光)写数据有效字节 $\dagger$ \\
        \textbf{\textcolor{red}{WLAST}} & Master & 1 & 写数据最后一个 \\
        WVALID & Master & 1 & 写数据有效 \\
        WREADY & Slave & 1 & 写数据准备好 \\
        \bottomrule
    \end{tabular}
\end{table}
$\dagger$ 例如数据:32'H1234\_5678 和掩码:4'BO111 代表:12 无效,其余有效。 WSTRB[n] correspondsto WDATA[(8 × n) + 7:(8 × n)].

\subsection{写响应}
\begin{table}[H]
    \centering
    \caption{写响应通道信号|Write response channel signals}
    \label{tab:write_response_channel_signals}
    \begin{tabular}{cccc}
        \toprule
        \textbf{通道}&  \textbf{来源} & \textbf{宽度} & \textbf{描述} \\
        \midrule
        BID[3:0] & Slave & 4 & 写响应ID \\
        \textbf{\textcolor{red}{BRESP[1:0]}} & Slave & 2 & 写响应状态 $\dagger$ \\
        BVALID & Slave & 1 & 写响应有效 \\
        BREADY & Master & 1 & 写响应准备好 \\
        \bottomrule
    \end{tabular}
\end{table}
$\dagger$ 写响应状态:00: OKAY, 01: EXOKAY, 10: SLVERR, 11: DECERR。

\subsection{读地址}
几乎完全和\ref{tab:write_address_channel_signals}一样。

\subsection{读数据}

把RESP信号和写数据信号并在一起了,功能类似写信号。

\subsection{低功耗接口信号}

\begin{table}[H]
    \centering
    \caption{低功耗接口信号(Low Power Interface Signals)}
    \label{tab:low_power_interface_signals}
    \begin{tabular}{cccc}
        \toprule
        \textbf{信号名称}  & \textbf{来源}             & \textbf{描述} \\
        \midrule
        CSYSREQ        & 时钟控制器(Clock controller) & SYS low-pow Request\\
        CSYSACK        & 外设(Peripheral device)      & ACK \\
        CACTIVE        & 外设(Peripheral device)      & 时钟活跃指示。$\dagger$ \\
        \bottomrule
    \end{tabular}
\end{table}
$\dagger$ 1=需要时钟,0=不需要时钟。表示外设是否需要其时钟信号。

\section{AXI RAM Master 实现}



\subsection{接口与参数定义}

AXI Master 模块的关键参数如下所示:

\begin{lstlisting}[columns=fullflexible,frame=shadowbox]
module axi_full_master #(
    parameter C_M_AXI_ID_WIDTH     = 1,
    parameter C_M_AXI_ADDR_WIDTH   = 32,
    parameter C_M_AXI_DATA_WIDTH   = 32,
    parameter C_M_TARGET_SLAVE_BASE_ADDR = 32'h00000000
)(
    input  wire                        M_AXI_ACLK,
    input  wire                        M_AXI_ARESETN,
\end{lstlisting}

Master 的接口包括五条 AXI 通道及用户控制接口,用户接口用于控制:操作类型(读/写)、启动信号、访问起始地址、突发长度、写入数据通道、读出数据通道

这使得 Master 能够作为上层模块(CPU 模拟器或测试逻辑)的直接驱动对象。

\subsection{二进制位宽计算函数}

AXI 协议中的 AWSIZE/ARSIZE 字段需要根据数据宽度自动计算。
使用如下函数:

\begin{lstlisting}[columns=fullflexible,frame=shadowbox]
function integer clogb2;
    input integer number;
    integer i;
begin
    clogb2 = 0;
    for(i = number-1; i > 0; i = i >> 1)
        clogb2 = clogb2 + 1;
end
endfunction
\end{lstlisting}

例如10 最后算出的位宽就是2。

\subsection{Master 内部寄存器与信号}

Master 的关键信号包括:

\begin{itemize}
    \item 地址锁存寄存器(读/写独立)
    \item 长度计数器
    \item 当前突发类型
    \item 写数据 beat 计数器
    \item 状态机状态寄存器
    \item 用户侧 busy/done/error 状态
\end{itemize}

示例:

\begin{lstlisting}[columns=fullflexible,frame=shadowbox]
reg [C_M_AXI_ADDR_WIDTH-1:0] addr_reg;
reg [7:0]                    len_reg;
reg                          rw_reg;
reg [7:0]                    wbeat_cnt;
reg                          error_reg;
reg                          done_reg;
\end{lstlisting}

\subsection{协议状态机设计/实现}

AXI Master 包含两套状态机:

\begin{enumerate}
    \item 写事务状态机:
    \begin{itemize}
        \item 写地址阶段(AW)
        \item 写数据阶段(W)
        \item 写响应阶段(B)
    \end{itemize}
    \item 读事务状态机:
    \begin{itemize}
        \item 读地址阶段(AR)
        \item 读数据阶段(R)
    \end{itemize}
\end{enumerate}

写状态机示例:

\begin{lstlisting}[columns=fullflexible,frame=shadowbox]
localparam ST_IDLE = 3'd0,
           ST_AW   = 3'd1,
           ST_W    = 3'd2,
           ST_B    = 3'd3,
           ST_DONE = 3'd4;
\end{lstlisting}

Master 仅在用户发出 start 信号时进入事务,并在 B/R 通道结束后回到空闲状态。

写地址的控制例:

\begin{lstlisting}[columns=fullflexible,frame=shadowbox]
always @(posedge M_AXI_ACLK) begin
    if (!M_AXI_ARESETN)
        M_AXI_AWVALID <= 1'b0;
    else if (state == ST_AW)
        M_AXI_AWVALID <= 1'b1;
    else if (M_AXI_AWVALID && M_AXI_AWREADY)
        M_AXI_AWVALID <= 1'b0;
end
\end{lstlisting}

读事务逻辑与其类似,但对应使用 AR/R 通道。

\subsection{Master 写数据逻辑示例}

写数据通道通过用户数据驱动:

\begin{lstlisting}[columns=fullflexible,frame=shadowbox]
assign M_AXI_WDATA  = user_wdata;
assign M_AXI_WVALID = (state == ST_W) && user_wvalid;
assign M_AXI_WLAST  = (wbeat_cnt == len_reg - 1);
\end{lstlisting}

写响应处理:

\begin{lstlisting}[columns=fullflexible,frame=shadowbox]
if (M_AXI_BVALID && M_AXI_BREADY) begin
    if (M_AXI_BRESP != 2'b00) error_reg <= 1'b1;
end
\end{lstlisting}

Master 将异常状态反馈到 user\_error。

\section{AXI RAM Slave 实现}

\subsection{目标效果}

我实现的 Slave 做到了:

\begin{itemize}
    \item 支持连续突发(INCR)、固定(FIXED)、回绕(WRAP)
    \item 支持字节写使能(WSTRB)
    \item 支持读写并发
    \item 可配置 RAM 深度
\end{itemize}

Slave 模块参数如下:

\begin{lstlisting}[columns=fullflexible,frame=shadowbox]
parameter C_S_RAM_DEPTH = 256;
reg [C_S_AXI_DATA_WIDTH-1:0] ram [0:C_S_RAM_DEPTH-1];
\end{lstlisting}

\subsection{地址递增函数实现}

支持三种突发模式:

\begin{lstlisting}[columns=fullflexible,frame=shadowbox]
function [C_S_AXI_ADDR_WIDTH-1:0] axi_next_addr;
    input [C_S_AXI_ADDR_WIDTH-1:0] addr;
    input [1:0] burst;
    input [2:0] size;
    input [7:0] len;
begin
    integer inc = (1 << size);
    integer bytes = inc * (len + 1);

    case (burst)
        2'b00: axi_next_addr = addr;            // FIXED
        2'b01: axi_next_addr = addr + inc;      // INCR
        2'b10: begin                            // WRAP
            reg [31:0] base = addr & ~(bytes-1);
            axi_next_addr = base |
               ((addr + inc) & (bytes-1));
        end
        default: axi_next_addr = addr + inc;
    endcase
end
endfunction
\end{lstlisting}

\subsection{写通道逻辑}

写地址握手:

\begin{lstlisting}[columns=fullflexible,frame=shadowbox]
assign S_AXI_AWREADY = ~wr_active;
always @(posedge S_AXI_ACLK) begin
    if (aw_hs) begin
        wr_active <= 1;
        wr_addr_reg <= S_AXI_AWADDR;
        wr_len_reg  <= S_AXI_AWLEN;
    end
end
\end{lstlisting}

写数据:

\begin{lstlisting}[columns=fullflexible,frame=shadowbox]
if (wr_active && w_hs) begin
    integer idx = wr_addr_reg[ADDR_LSB +: RAM_ADDR_WIDTH];
    for (i=0;i<C_S_AXI_DATA_WIDTH/8;i=i+1)
        if (S_AXI_WSTRB[i])
            ram[idx][8*i +: 8] <= S_AXI_WDATA[8*i +: 8];
end
\end{lstlisting}

\subsection{读通道逻辑}

读地址握手:

\begin{lstlisting}[columns=fullflexible,frame=shadowbox]
assign S_AXI_ARREADY = ~rd_active;
if (ar_hs) begin
    rd_active <= 1;
    rd_addr_reg <= S_AXI_ARADDR;
    rd_len_reg  <= S_AXI_ARLEN;
end
\end{lstlisting}

读数据产生:

\begin{lstlisting}[columns=fullflexible,frame=shadowbox]
integer idx_r = rd_addr_reg[ADDR_LSB +: RAM_ADDR_WIDTH];
rdata_reg <= ram[idx_r];
rvalid_reg <= 1;
rlast_reg <= (rd_cnt == rd_len_reg);
\end{lstlisting}

Slave 按协议连续输出数据直到最后一个 beat。

\section{AXI总线与CPU的集成}

\subsection{系统架构设计/实现}

在五级流水CPU中,访存操作主要发生在两个阶段:

\begin{enumerate}
    \item \textbf{取指阶段(IF)}:从指令存储器读取指令
    \item \textbf{访存阶段(MEM)}:执行 load/store 指令,访问数据存储器
\end{enumerate}

为此,我采用了了\textbf{双AXI Master}架构:(实际上就是替换掉之前的ROM和RAM,都改成RAM)

\begin{itemize}
    \item \textbf{指令AXI Master}:专门用于取指,连接到指令RAM Slave
    \item \textbf{数据AXI Master}:用于load/store操作,连接到数据RAM Slave
\end{itemize}

顶层模块接口修改如下:

\begin{lstlisting}[columns=fullflexible,frame=shadowbox]
module pipeline_cpu(
    input clk,
    input resetn,
    
    // 指令AXI Master接口
    output [31:0] M_AXI_INSTR_ARADDR,
    output [7:0]  M_AXI_INSTR_ARLEN,
    output        M_AXI_INSTR_ARVALID,
    input         M_AXI_INSTR_ARREADY,
    input  [31:0] M_AXI_INSTR_RDATA,
    input         M_AXI_INSTR_RVALID,
    output        M_AXI_INSTR_RREADY,
    // ... 其他AXI信号
    
    // 数据AXI Master接口
    output [31:0] M_AXI_DATA_AWADDR,
    output [31:0] M_AXI_DATA_ARADDR,
    // ... 其他AXI信号
);
\end{lstlisting}

\subsection{取指模块(Fetch)的AXI接口改造}

原始的fetch模块直接连接inst\_rom,现在需要通过AXI总线获取指令。主要修改包括:

\subsubsection{状态机设计}

设计了一个4状态的取指状态机:

\begin{lstlisting}[columns=fullflexible,frame=shadowbox]
parameter IDLE    = 2'b00,  // 空闲
          REQUEST = 2'b01,  // 发起AXI请求
          PENDING = 2'b10,  // 等待AXI响应
          DONE    = 2'b11;  // 指令就绪
\end{lstlisting}

状态转移逻辑:

\begin{lstlisting}[columns=fullflexible,frame=shadowbox]
always @(*) begin
    case(current_state)
        IDLE: begin
            if (IF_valid && !axi_busy)
                next_state = REQUEST;
        end
        REQUEST: begin
            next_state = PENDING;
        end
        PENDING: begin
            if (axi_done)
                next_state = DONE;
        end
        DONE: begin
            if (next_fetch)  // 流水线接受指令
                next_state = IDLE;
        end
    endcase
end
\end{lstlisting}

\subsubsection{与流水线的握手机制}

关键信号说明:
\begin{itemize}
    \item \texttt{axi\_start}:向AXI Master发起读请求(1周期脉冲)
    \item \texttt{axi\_done}:AXI读事务完成,指令有效
    \item \texttt{IF\_over}:通知下一级指令准备就绪
    \item \texttt{next\_fetch}:下一级允许接收新指令
\end{itemize}

\subsection{访存模块(MEM)的AXI接口改造}

MEM模块需要处理两类操作:

\begin{enumerate}
    \item \textbf{非访存指令}(如算术运算):直接通过,\texttt{MEM\_over}立即置1
    \item \textbf{访存指令}(load/store):发起AXI事务,等待\texttt{axi\_done}
\end{enumerate}

关键逻辑:

\begin{lstlisting}[columns=fullflexible,frame=shadowbox]
// MEM完成信号:区分访存/非访存指令
assign MEM_over = (inst_load | inst_store) ? 
                   axi_done : MEM_valid;

// 防止重复发起AXI请求
reg axi_started;
always @(posedge clk) begin
    if (MEM_allow_in)
        axi_started <= 1'b0;
    
    if (do_load && !axi_started) begin
        axi_start <= 1'b1;
        axi_started <= 1'b1;
    end
end
\end{lstlisting}

\section{调试过程与问题解决}

\subsection{问题1:取指阶段无法获取指令}

\subsubsection{现象}
仿真波形显示:
\begin{itemize}
    \item \texttt{IF\_pc} 始终为 \texttt{0x00000034}(初始值)
    \item \texttt{IF\_inst} 始终为 0
    \item \texttt{IF\_over} 始终为 0
    \item 流水线完全停滞
\end{itemize}

\subsubsection{根本原因}

PC更新条件设置错误:

\begin{lstlisting}[columns=fullflexible,frame=shadowbox]
// 错误的逻辑(问题代码)
always @(posedge clk) begin
    if (next_fetch && axi_done)  // 两个信号很少同拍
        pc <= next_pc;
end
\end{lstlisting}

\textbf{分析}:
\begin{itemize}
    \item \texttt{axi\_done} 是AXI读完成信号(脉冲)
    \item \texttt{next\_fetch} 是流水线允许信号,依赖于\texttt{IF\_over}
    \item \texttt{IF\_over} 又依赖于\texttt{axi\_done}
    \item 结果:两信号时序错位,永远无法同时为1
\end{itemize}

\subsubsection{解决方案}

修改PC更新逻辑为"上一条指令完成且允许取下一条":

\begin{lstlisting}[columns=fullflexible,frame=shadowbox]
// 正确的逻辑
always @(posedge clk) begin
    if (!resetn)
        pc <= `STARTADDR;
    else if (next_fetch && IF_over)  // 指令完成+允许
        pc <= next_pc;
end
\end{lstlisting}

\subsection{问题2:指令无法传播到后续流水级}

\subsubsection{现象}

波形显示:
\begin{itemize}
    \item 第一条指令成功取到:\texttt{IF\_inst = 0x24010001}
    \item \texttt{IF\_over = 1}(正确)
    \item 但 \texttt{ID\_valid} 始终为0,指令未进入ID级
\end{itemize}

\subsubsection{根本原因}

\texttt{IF\_over}是瞬时脉冲信号,如果ID级当拍没有接受,信号就消失了:

\begin{lstlisting}[columns=fullflexible,frame=shadowbox]
// 问题代码:IF_over只维持1个周期
always @(posedge clk)
    IF_over <= axi_done;
\end{lstlisting}

\subsubsection{解决方案}

改为\textbf{握手保持}机制:

\begin{lstlisting}[columns=fullflexible,frame=shadowbox]
// 修正:保持IF_over直到被接受
always @(posedge clk) begin
    if (!resetn)
        IF_over <= 1'b0;
    else if (axi_done)     // AXI完成时置位
        IF_over <= 1'b1;
    else if (next_fetch)   // 被接受时清零
        IF_over <= 1'b0;
end
\end{lstlisting}

\subsection{问题3:ID级缺少时钟信号}

\subsubsection{现象}

综合时报错:

\begin{verbatim}
ERROR: [VRFC 10-2989] 'clk' is not declared 
       [decode.v:332]
\end{verbatim}

\subsubsection{原因}

为了解决时序对齐问题,需要在decode模块中打拍\texttt{IF\_over}信号:

\begin{lstlisting}[columns=fullflexible,frame=shadowbox]
// decode.v 中新增的打拍逻辑
reg IF_over_d;
always @(posedge clk) begin
    if (!resetn)
        IF_over_d <= 1'b0;
    else
        IF_over_d <= IF_over;
end
\end{lstlisting}

但decode模块端口没有\texttt{clk}输入。

\subsubsection{解决方案}

修改decode模块接口和实例化:

\begin{lstlisting}[columns=fullflexible,frame=shadowbox]
// decode.v 端口增加
module decode(
    input clk,        // 新增
    input resetn,     // 新增
    input ID_valid,
    // ... 其他端口
);

// pipeline_cpu.v 实例化时连接
decode ID_module(
    .clk(clk),        // 新增连接
    .resetn(resetn),  // 新增连接
    .ID_valid(ID_valid),
    // ... 其他连接
);
\end{lstlisting}

\subsection{问题4:非访存指令卡在MEM级}

\subsubsection{现象}

波形分析:

\begin{verbatim}
Time=205000: addiu指令进入ID级,ID_over=1
Time=215000: addiu指令进入EXE级
Time=225000: addiu指令进入MEM级
Time=235000: MEM_valid=1, MEM_allow=0 ← 卡住!
             后续所有指令都停在EXE级之前
\end{verbatim}

\subsubsection{根本原因}

MEM模块的完成信号设置不当:

\begin{lstlisting}[columns=fullflexible,frame=shadowbox]
// 错误逻辑:所有指令都等axi_done
assign MEM_over = axi_done;
\end{lstlisting}

\textbf{分析}:\texttt{addiu}是算术指令,不需要访存,但MEM模块仍然等待\texttt{axi\_done}。由于没有发起AXI事务,\texttt{axi\_done}永远不会来,导致死锁。

\subsubsection{解决方案}

区分访存和非访存指令:

\begin{lstlisting}[columns=fullflexible,frame=shadowbox]
// 正确逻辑
assign MEM_over = (inst_load | inst_store) ? 
                   axi_done :      // 访存指令等AXI
                   MEM_valid;      // 非访存立即完成
\end{lstlisting}

\subsection{问题5:读取到错误的指令数据}

\subsubsection{现象}

调试信息显示:

\begin{verbatim}
Time=125000 FETCH: Requesting PC=00000000
Time=155000 AXI_SLAVE: AR handshake, ARADDR=00000000
Time=165000 AXI_SLAVE READ: addr=0x00, idx=0, 
            data=24010001 (正确)
Time=175000 AXI_SLAVE READ: addr=0x04, idx=1, 
            data=00011100 (错误!多读一次)
Time=195000 CPU收到: Inst=00011100 ← 错误数据
\end{verbatim}

指令RAM内容:
\begin{itemize}
    \item \texttt{ram[0] = 0x24010001}(正确)
    \item \texttt{ram[1] = 0x00011100}
\end{itemize}

CPU应该读到\texttt{ram[0]},但实际读到了\texttt{ram[1]}。

\subsubsection{根本原因}

AXI Slave的读数据产生逻辑存在bug:

\begin{lstlisting}[columns=fullflexible,frame=shadowbox]
// 问题代码
if (rd_active && (!rvalid_reg || r_hs)) begin
    // 没有检查是否已发完所有beat!
    idx_r = (rd_addr_reg >> 2);
    rdata_reg <= ram[idx_r];
    rd_cnt <= rd_cnt + 1;
    rd_addr_reg <= rd_addr_reg + 4;  // 地址递增
end
\end{lstlisting}

\textbf{问题分析}:

当\texttt{ARLEN=0}(只读1个beat)时:
\begin{enumerate}
    \item 第1个周期:\texttt{rd\_cnt=0},读\texttt{ram[0]},\texttt{rd\_cnt}变成1,地址+4
    \item 第2个周期:条件 \texttt{(!rvalid\_reg || r\_hs)} 仍然满足,又读了\texttt{ram[1]}
    \item CPU最终收到第2次读出的错误数据
\end{enumerate}

\subsubsection{解决方案}

增加beat计数检查,防止超读:

\begin{lstlisting}[columns=fullflexible,frame=shadowbox]
// 修正后的逻辑
if (rd_active && (!rvalid_reg || r_hs) 
    && (rd_cnt <= rd_len_reg)) begin  // 增加检查
    
    idx_r = (rd_addr_reg >> 2);
    rdata_reg <= ram[idx_r];
    rvalid_reg <= 1'b1;
    rlast_reg <= (rd_cnt == rd_len_reg);
    
    rd_cnt <= rd_cnt + 1;
    
    // 只在非最后一拍时递增地址
    if (rd_cnt < rd_len_reg)
        rd_addr_reg <= rd_addr_reg + 4;
end

// 最后一拍握手后结束事务
if (rvalid_reg && rlast_reg && r_hs)
    rd_active <= 1'b0;
\end{lstlisting}


\section{验证结果}

\subsection{整体功能验证}

最终波形显示CPU正常工作:


\begin{figure}[H]
    \centering
    \includegraphics[width=\textwidth]{img/整体流程.png}
    \caption{整体流程}
\end{figure}

\begin{verbatim}[caption=调试的时候增加的输出信息]
    Time=195000: PC=0x00, Inst=0x24010001 (addiu $1,$0,1)
    Time=295000: PC=0x04, Inst=0x00011100 (sll $2,$1,4)
    Time=395000: PC=0x08, Inst=0x00411821 (addu $3,$2,$1)
    Time=495000: PC=0x0C, Inst=0x00022082 (srl $4,$2,2)
    ...
\end{verbatim}


可以看到之前是两周期的变成了7个周期,其具体为:

\begin{itemize}
    \item AXI Master状态转换:2周期
    \item AXI握手:1周期
    \item AXI Slave读RAM:1周期
    \item 数据传输:1周期
    \item 流水线锁存:2周期
\end{itemize}

流水线的效率反而更低了。但可能我认为扩展性应该会好一些,因为之后的cache等等也要加上去。


\subsection{cpu\_5\_valid信号变化}

下面我想分析我自己写的这个信号:\texttt{cpu\_5\_valid} 是一个32位的流水线状态指示信号,其编码方式如下。

\begin{lstlisting}[columns=fullflexible,frame=shadowbox]
assign cpu_5_valid = {12'd0,           // [31:20] 保留位
                      {4{IF_valid}},   // [19:16] IF级有效
                      {4{ID_valid}},   // [15:12] ID级有效
                      {4{EXE_valid}},  // [11:8]  EXE级有效
                      {4{MEM_valid}},  // [7:4]   MEM级有效
                      {4{WB_valid}}};  // [3:0]   WB级有效
\end{lstlisting}

从波形图中可以清晰观察到流水线的\textbf{充填过程}(Pipeline Filling):

\begin{figure}[H]
    \centering
    \includegraphics[width=1.0\textwidth]{img/cp_valid.png}
    \caption{流水线充填过程波形图}
\end{figure}

\subsubsection{信号变化分析}

\textbf{1. 初始状态(t=290ns-300ns)}
\begin{itemize}
    \item \texttt{cpu\_5\_valid = 0x000F0000}
    \item 解析:仅IF级有效(\texttt{IF\_valid=F},其余为0)
    \item 说明:第一条指令进入取指阶段,后续流水级均空闲
\end{itemize}

\textbf{2. 第一条指令推进(t=300ns-310ns)}
\begin{itemize}
    \item \texttt{cpu\_5\_valid = 0x000FF000}
    \item 解析:IF和ID级同时有效
    \item 说明:第一条指令(\texttt{PC=0x04, Inst=0x24010001})进入ID级译码,同时第二条指令进入IF级
\end{itemize}

\textbf{3. 流水线继续充填(t=310ns-330ns)}
\begin{itemize}
    \item \texttt{cpu\_5\_valid = 0x000F0F00}(t≈315ns)
    \item 解析:IF和EXE级有效,ID级已空
    \item 说明:第一条指令进入执行阶段,流水线出现\textbf{气泡}(bubble)
    
    \item \texttt{cpu\_5\_valid = 0x000F00F0}(t≈325ns)
    \item 解析:IF和MEM级有效
    \item 说明:第一条指令进入访存阶段
    
    \item \texttt{cpu\_5\_valid = 0x000F000F}(t≈335ns)
    \item 解析:IF和WB级有效
    \item 说明:第一条指令进入写回阶段
\end{itemize}

\textbf{4. 稳定运行阶段(t>340ns)}
\begin{itemize}
    \item \texttt{cpu\_5\_valid = 0x000F0000}(周期性变化)
    \item 说明:由于AXI访存延迟,流水线未能达到理想的5级满载状态
    \item 观察到MEM级的PC停留在\texttt{0x00000004}较长时间(高亮显示)
    \item 表明该指令在MEM阶段等待AXI事务完成
\end{itemize}

\subsubsection{关键现象解读}

\textbf{流水线气泡(Pipeline Bubble)}

从\texttt{cpu\_5\_valid}的跳变可以看出,流水线并非连续满载,而是存在明显的气泡:
\begin{itemize}
    \item 从\texttt{0x000FF000}(IF+ID)直接跳到\texttt{0x000F0F00}(IF+EXE),跳过了完整的3级同时有效状态
    \item 原因:\textbf{AXI取指延迟}导致下一条指令无法及时进入流水线
\end{itemize}

\textbf{MEM级停顿}

波形中MEM\_pc长时间保持\texttt{0x00000004}:
\begin{itemize}
    \item 说明该指令为\textbf{访存指令}(load/store)
    \item MEM级等待数据AXI Master完成总线事务
    \item 此时\texttt{MEM\_allow=0},阻塞了后续指令的推进
\end{itemize}

\newpage

\subsection{AXI指令总线信号分析}

从波形图中可以观察到指令AXI Master的关键信号行为,这些信号反映了CPU取指过程中的AXI总线交互细节。

\begin{figure}[H]
    \centering
    \includegraphics[width=\textwidth]{img/AXI详细信号.png}
    \caption{AXI详细信号}
\end{figure}

\subsubsection{M\_AXI\_INSTR\_AWADDR[31:0] - 读地址信号}

\textbf{信号特征:}
\begin{itemize}
    \item 波形显示地址按\textbf{字对齐}递增:\texttt{0x04 → 0x08 → 0x0C → 0x10}
    \item 每次取指发起一个新的读地址请求
    \item 地址递增步长为4字节(32位指令宽度)
\end{itemize}

\textbf{时序关系:}
\begin{itemize}
    \item 地址变化领先于指令数据到达约50-70ns
    \item 例如:\texttt{AWADDR=0x08}时刻(t≈350ns),对应的指令\texttt{0x00011100}在约400ns时出现在\texttt{IF\_inst}
    \item 说明从地址发出到数据返回经历了完整的AXI读事务周期
\end{itemize}

\textbf{与流水线PC的对应关系:}
\begin{itemize}
    \item \texttt{M\_AXI\_INSTR\_AWADDR}总是与\texttt{IF\_pc}保持同步
    \item 表明Fetch模块的PC值正确传递给AXI Master
    \item 地址连续性验证了流水线取指的正确性
\end{itemize}

\subsubsection{M\_AXI\_INSTR\_RVALID - 读数据有效信号}

\textbf{信号特征:}
\begin{itemize}
    \item 表现为\textbf{周期性脉冲},每个脉冲对应一次成功的指令读取
    \item 脉冲宽度约1-2个时钟周期
    \item 脉冲间隔约100ns,与取指周期一致
\end{itemize}

\textbf{AXI握手分析:}
\begin{lstlisting}[columns=fullflexible,frame=shadowbox]
// AXI读数据通道握手条件
读数据传输成功 = RVALID & RREADY
\end{lstlisting}

从波形可以看出:
\begin{itemize}
    \item \texttt{RVALID}拉高后,需要等待\texttt{RREADY}同时为高才完成握手
    \item 每次\texttt{RVALID}脉冲后,\texttt{IF\_inst}立即更新为新的指令数据
    \item 例如:t≈300ns处\texttt{RVALID}有效,\texttt{IF\_inst}更新为\texttt{0x24010001}
\end{itemize}

\textbf{与cpu\_5\_valid的关联:}
\begin{itemize}
    \item \texttt{RVALID}有效时,\texttt{cpu\_5\_valid}中IF级标志保持有效
    \item 说明取指模块正确响应了AXI Slave返回的数据
    \item 验证了\texttt{axi\_done}信号与\texttt{RVALID}的逻辑关系
\end{itemize}

\subsubsection{M\_AXI\_INSTR\_RREADY - 读数据准备信号}

\textbf{信号特征:}
\begin{itemize}
    \item 与\texttt{RVALID}时序\textbf{紧密配合},形成握手对
    \item 波形显示\texttt{RREADY}几乎总是在\texttt{RVALID}之前或同时拉高
    \item 说明CPU端(Master)始终准备好接收指令数据
\end{itemize}

\textbf{实现分析:}

在实现中,\texttt{RREADY}被设置为常高:

\begin{lstlisting}[columns=fullflexible,frame=shadowbox]
// AXI Master 中的实现
.user_rready(1'b1)  // 指令读取通道永远ready
\end{lstlisting}


\textbf{握手时序验证:}
\begin{itemize}
    \item 从波形看,每次\texttt{RVALID \& RREADY}同时为高时
    \item \texttt{IF\_inst}准确更新,PC递增
    \item 流水线推进,\texttt{cpu\_5\_valid}变化
    \item 完整验证了AXI读数据通道的正确性
\end{itemize}

\subsubsection{M\_AXI\_INSTR\_BRESP[1:0] - 写响应信号}

\textbf{信号特征:}
\begin{itemize}
    \item 波形显示该信号始终为\texttt{0}(即\texttt{2'b00 = OKAY})
    \item 在整个仿真过程中无变化
\end{itemize}

\textbf{原因分析:}

指令AXI接口是\textbf{只读通道},因此写响应信号不会被使用:

\begin{lstlisting}[columns=fullflexible,frame=shadowbox]
// 指令AXI Master配置
.user_rw(1'b1),  // 固定为读操作
\end{lstlisting}

\begin{itemize}
    \item \texttt{BRESP}属于AXI\textbf{写响应通道}(B通道)
    \item 取指只使用\textbf{读地址通道}(AR)和\textbf{读数据通道}(R)
    \item 写地址(AW)、写数据(W)、写响应(B)三通道在指令总线中未激活
    \item \texttt{BRESP=0}表示默认的OKAY状态,符合协议要求
\end{itemize}

\textbf{对比:数据AXI总线}

与之相对应,数据AXI Master需要支持store指令,因此:
\begin{itemize}
    \item 数据总线的\texttt{BRESP}会在store操作时产生有效响应
    \item 波形中如果有store指令执行,会看到\texttt{M\_AXI\_DATA\_BRESP}的变化
    \item 验证了指令总线和数据总线的\textbf{功能分离}设计/实现
\end{itemize}

\subsubsection{AXI信号时序总结}

通过对这四个关键信号的分析,可以总结出完整的AXI取指时序流程:

\begin{enumerate}
    \item \textbf{地址阶段}(AR通道):
    \begin{itemize}
        \item Fetch模块发起请求,\texttt{AWADDR}更新为当前PC
        \item AXI Master进入读地址状态,发送\texttt{ARVALID}(波形中未显示)
        \item Slave响应\texttt{ARREADY},完成地址握手
    \end{itemize}
    
    \item \textbf{数据阶段}(R通道):
    \begin{itemize}
        \item 经过若干周期延迟(Slave从RAM读取)
        \item \texttt{RVALID}拉高,指令数据同步输出
        \item CPU端\texttt{RREADY}已准备好,立即完成握手
        \item \texttt{IF\_inst}更新,\texttt{axi\_done}有效
    \end{itemize}
    
    \item \textbf{流水线推进}:
    \begin{itemize}
        \item \texttt{IF\_over}置1,通知下一级
        \item PC递增+4,准备下一次取指
        \item \texttt{cpu\_5\_valid}变化,指令进入ID级
    \end{itemize}
\end{enumerate}

从波形的AXI信号可以清楚看出,每个取指周期严格遵循上述流程,验证了AXI总线与CPU流水线的正确集成。

\section{总结}

\subsection{实验成果}

\begin{enumerate}
    \item 成功实现了符合AXI4协议的Master和Slave模块
    \item 将AXI总线集成到五级流水CPU中
    \item 解决了多个复杂的时序和握手问题
    \item 实现了完整的指令取指和数据访存通路
\end{enumerate}

\subsection{关键技术点}

\begin{itemize}
    \item \textbf{握手协议}:AXI的VALID/READY双向握手机制
    \item \textbf{突发传输}:理解ARLEN/AWLEN与beat计数的关系
    \item \textbf{流水线控制}:over/allow信号的协调
    \item \textbf{状态机实现}:避免死锁,保证事务完整性
\end{itemize}



\label{sec:appendix}
\section*{附录(补充说明)}
\subsection{编址方法}

\begin{figure}[H]
    \centering
    \includegraphics[width=\textwidth]{img/硬编址.png}
    \caption{硬编址}
\end{figure}

我把代码写死在了slave模块的初始模块中,最开始采取AI建议的一种加载方式,发现不行。这样的一个缺点是,因为我mem.v实例化的也是这个模块,导致了mem RAM初始的时候不是全0,不过这个不影响正常运行。



\label{LastPage}
\end{document}
