\documentclass[a4paper]{article}
\input{style/ch_xelatex.tex}
% "define" Scala
\usepackage[T1]{fontenc}  
\usepackage[scaled=0.82]{beramono}  
\usepackage{microtype} 

\sbox0{\small\ttfamily A}
\edef\mybasewidth{\the\wd0 }

\lstdefinelanguage{scala}{
  morekeywords={abstract,case,catch,class,def,%
    do,else,extends,false,final,finally,%
    for,if,implicit,import,match,mixin,%
    new,null,object,override,package,%
    private,protected,requires,return,sealed,%
    super,this,throw,trait,true,try,%
    type,val,var,while,with,yield},
  sensitive=true,
  morecomment=[l]{//},
  morecomment=[n]{/*}{*/},
  morestring=[b]",
  morestring=[b]',
  morestring=[b]"""
}

\usepackage{color}
\definecolor{dkgreen}{rgb}{0,0.6,0}
\definecolor{gray}{rgb}{0.5,0.5,0.5}
\definecolor{mauve}{rgb}{0.58,0,0.82}

% Default settings for code listings
\lstset{frame=tb,
  language=scala,
  aboveskip=3mm,
  belowskip=3mm,
  showstringspaces=false,
  columns=fixed, % basewidth=\mybasewidth,
  basicstyle={\small\ttfamily},
  numbers=none,
  numberstyle=\footnotesize\color{gray},
  % identifierstyle=\color{red},
  keywordstyle=\color{blue},
  commentstyle=\color{dkgreen},
  stringstyle=\color{mauve},
  frame=single,
  breaklines=true,
  breakatwhitespace=true,
  procnamekeys={def, val, var, class, trait, object, extends},
  procnamestyle=\ttfamily\color{red},
  tabsize=2
}

\lstnewenvironment{scala}[1][]
{\lstset{language=scala,#1}}
{}
\lstnewenvironment{cpp}[1][]
{\lstset{language=C++,#1}}
{}
\lstnewenvironment{bash}[1][]
{\lstset{language=bash,#1}}
{}
\lstnewenvironment{verilog}[1][]
{\lstset{language=verilog,#1}}
{}



\graphicspath{ {images/} }
\usepackage{ctex}
\usepackage{graphicx}
\usepackage{color,framed}%文本框
\usepackage{listings}
\usepackage{caption}

\usepackage{hyperref}
\hypersetup{hidelinks,
	colorlinks=true,
	allcolors=black,
	pdfstartview=Fit,
	breaklinks=true}


\lstdefinestyle{mystyle}{
  keywordstyle=\color{orange},  % 设置关键词颜色为橙色
 morekeywords={set, nmap, msfconsole,exploit},           % 仅将 set 设置为橙色
}

\usepackage{amssymb}
\usepackage{enumerate}
\usepackage{xcolor}
\usepackage{bm} 
\usepackage{lastpage}%获得总页数
\usepackage{fancyhdr}
\usepackage{tabularx}  
\usepackage{geometry}
\usepackage{minted}
\usepackage{graphics}
\usepackage{subfigure}
\usepackage{float}
\usepackage{pdfpages}
\usepackage{pgfplots}
\pgfplotsset{width=10cm,compat=1.9}
\usepackage{multirow}
\usepackage{footnote}
\usepackage{booktabs}
\usepackage{listings}

%-----------------------伪代码------------------
\usepackage{algorithm}  
\usepackage{algorithmicx}  
\usepackage{algpseudocode}  
\floatname{algorithm}{Algorithm}  
\renewcommand{\algorithmicrequire}{\textbf{Input:}}  
\renewcommand{\algorithmicensure}{\textbf{Output:}} 
\usepackage{lipsum}  
\makeatletter
\newenvironment{breakablealgorithm}
  {% \begin{breakablealgorithm}
  \begin{center}
     \refstepcounter{algorithm}% New algorithm
     \hrule height.8pt depth0pt \kern2pt% \@fs@pre for \@fs@ruled
     \renewcommand{\caption}[2][\relax]{% Make a new \caption
      {\raggedright\textbf{\ALG@name~\thealgorithm} ##2\par}%
      \ifx\relax##1\relax % #1 is \relax
         \addcontentsline{loa}{algorithm}{\protect\numberline{\thealgorithm}##2}%
      \else % #1 is not \relax
         \addcontentsline{loa}{algorithm}{\protect\numberline{\thealgorithm}##1}%
      \fi
      \kern2pt\hrule\kern2pt
     }
  }{% \end{breakablealgorithm}
     \kern2pt\hrule\relax% \@fs@post for \@fs@ruled
  \end{center}
  }
\makeatother
%------------------------代码-------------------
\RequirePackage{listings}
\RequirePackage{xcolor}
\definecolor{dkgreen}{rgb}{0,0.6,0}
\definecolor{gray}{rgb}{0.5,0.5,0.5}
\definecolor{mauve}{rgb}{0.58,0,0.82}
\lstset{
	numbers=left,  
	frame=tb,
	aboveskip=3mm,
	belowskip=3mm,
	showstringspaces=false,
	columns=flexible,
	framerule=1pt,
	rulecolor=\color{gray!35},
	backgroundcolor=\color{gray!5},
	basicstyle={\ttfamily},
	numberstyle=\tiny\color{gray},
	keywordstyle=\color{blue},
	commentstyle=\color{dkgreen},
	stringstyle=\color{mauve},
	breaklines=true,
	breakatwhitespace=true,
	tabsize=3,
}


%-------------------------页面边距--------------
\geometry{a4paper,left=2.3cm,right=2.3cm,top=2.7cm,bottom=2.7cm}
%-------------------------页眉页脚--------------
\usepackage{fancyhdr}
\pagestyle{fancy}
\lhead{\kaishu }
\chead{}
\rhead{{\CJKfontspec{simkai.ttf} }}
\lfoot{}
\cfoot{\thepage}
\rfoot{}
\renewcommand{\headrulewidth}{0pt}  
\renewcommand{\footrulewidth}{0pt}
\newcommand{\HRule}{\rule{\linewidth}{0.5mm}}
\newcommand{\HRulegrossa}{\rule{\linewidth}{1.2mm}}
\setlength{\textfloatsep}{10mm}
%--------------------文档内容--------------------
\renewcommand{\contentsname}{目\ 录}
\renewcommand{\appendixname}{附录}
\renewcommand{\appendixpagename}{附录}
\renewcommand{\refname}{参考文献} 
\renewcommand{\figurename}{图}
\renewcommand{\tablename}{表}
\renewcommand{\today}{\number\year 年 \number\month 月 \number\day 日}

\renewcommand {\thefigure}{\thesection{}.\arabic{figure}}%图片按章标号
\renewcommand{\figurename}{图}
\renewcommand{\contentsname}{目录}  
\cfoot{\thepage\ of \pageref{LastPage}}%当前页 of 总页数

\renewcommand{\abstractname}{\textbf{\Large 摘要}} % 调整摘要标题的字体大小
\renewcommand {\thefigure}{\thesection{}.\arabic{figure}}%图片按章标号


% 代码样式设置(Verilog)
\lstdefinelanguage{Verilog}{
    morekeywords={module, endmodule, input, output, wire, reg, always, assign, if, else, begin, end, posedge, negedge, parameter, localparam, case, default},
    sensitive=true,
    morecomment=[l]{//},
    morecomment=[n]{/*}{*/},
    morestring=[b]"
}

\lstset{
    language=Verilog,
    basicstyle=\ttfamily\small,
    keywordstyle=\color{blue}\bfseries,
    commentstyle=\color{dkgreen},
    stringstyle=\color{red},
    numbers=left,
    numberstyle=\tiny\color{gray},
    stepnumber=1,
    numbersep=5pt,
    backgroundcolor=\color{gray!10},
    frame=single,
    breaklines=true,
    breakatwhitespace=true,
    tabsize=4,
    showspaces=false,
    showstringspaces=false
}

\usepackage{booktabs}
\usepackage{geometry}
\geometry{left=2.5cm,right=2.5cm,top=3cm,bottom=3cm}

\begin{document}

\renewcommand{\figurename}{图}
\renewcommand{\contentsname}{目录}  
\cfoot{\thepage\ of \pageref{LastPage}}
\renewcommand{\abstractname}{\textbf{\Large 摘要}} 

\begin{center}
    \huge{\textbf{MIPS五级流水线CPU的CP0异常处理系统设计与实现}}
\end{center}

\begin{center}
    \textbf{姓名}:\underline{姓名} \quad
    \textbf{专业}:\underline{专业}
\end{center}

\tableofcontents

\vspace*{1cm}

\noindent{\Large\textbf{摘要}}

\vspace{1em}

\textbf{CP0模块实现部分:} 本文档详细描述了MIPS五级流水线CPU中协处理器0(CP0)的设计与实现。CP0是MIPS架构中用于系统控制和异常处理的关键模块,负责处理各种异常类型(如SYSCALL、BREAK、地址错误等)和中断(如定时器中断),并维护系统状态寄存器。本文档涵盖了CP0模块的总体架构、6个CP0寄存器的实现细节、异常处理流程和仲裁机制、4种异常类型的实现、总线排布和信号传递、CP0寄存器访问机制以及流水线控制信号等内容。

\textbf{软件异常处理程序部分:} 本章节预留,用于描述异常处理程序的编写方法、测试程序的设计思路等。具体内容待补充。

\vspace{1em}
\noindent\textbf{关键词:} MIPS、五级流水线、CP0、异常处理、中断、协处理器

\newpage

\section{实验目的}

\begin{enumerate}
    \item 理解MIPS架构中CP0(协处理器0)的作用和功能
    \item 掌握CP0寄存器的设计和实现方法
    \item 学习异常处理机制的设计与实现
    \item 理解中断处理流程和定时器中断的实现
    \item 掌握流水线中异常信号的传递和仲裁机制
\end{enumerate}

\section{实验要求}

\begin{enumerate}
    \item 实现CP0模块,包含STATUS、CAUSE、EPC、BADVADDR、COUNT、COMPARE等6个寄存器
    \item 实现SYSCALL、BREAK、地址错误(AdEL/AdES)、定时器中断等4种异常类型
    \item 实现异常仲裁机制,确定异常优先级
    \item 实现MTC0/MFC0指令,支持CP0寄存器访问
    \item 实现ERET指令,支持从异常处理返回
\end{enumerate}

\section{实验过程}

\subsection{CP0模块总体架构}

\subsubsection{模块接口设计}

CP0模块(\texttt{cp0.v})是异常处理系统的核心,其接口定义如下:

\begin{lstlisting}[caption=CP0模块接口定义]
module cp0(
    input             clk,       // 时钟
    input             resetn,    // 复位信号,低电平有效
    
    // 来自WB级的控制信号
    input             mtc0,      // MTC0指令标识
    input             mfc0,      // MFC0指令标识
    input      [ 7:0] cp0r_addr, // CP0寄存器地址 {寄存器号[4:0], 选择域[2:0]}
    input      [31:0] wdata,     // 写入CP0的数据
    
    // 异常相关信号
    input             syscall,   // SYSCALL指令标识
    input             eret,      // ERET指令标识
    input      [31:0] pc,        // 当前PC值(用于保存到EPC)
    input             wb_valid,  // WB级有效信号
    input             wb_over,   // WB级完成信号
    
    // 统一异常总线(来自WB的最终裁决)
    input             ex_valid_i,        // 异常有效
    input      [ 4:0] ex_code_i,         // 异常编码
    input             ex_bd_i,           // 延迟槽异常
    input      [31:0] ex_pc_i,           // 发生异常的PC
    input             badvaddr_valid_i,  // 错误地址有效
    input      [31:0] badvaddr_i,        // 错误地址
    
    // CP0寄存器读数据输出
    output     [31:0] cp0r_rdata,        // CP0寄存器读数据(用于MFC0)
    
    // 异常处理输出
    output            cancel,    // 取消流水线信号
    output            exc_valid, // 异常有效信号
    output     [31:0] exc_pc,    // 异常入口地址或ERET返回地址
    
    // 寄存器值输出(用于异常处理)
    output     [31:0] cp0r_status,       // STATUS寄存器值
    output     [31:0] cp0r_cause,         // CAUSE寄存器值
    output     [31:0] cp0r_epc,           // EPC寄存器值
    
    // 中断输出
    output            c0_int              // 中断有效信号
);
\end{lstlisting}

\subsubsection{设计原则}

CP0模块的设计遵循以下原则:

\begin{enumerate}
    \item \textbf{统一异常总线}:所有异常(包括SYSCALL、BREAK、地址错误、中断)都通过统一的异常总线(\texttt{ex\_valid\_i, ex\_code\_i}等)传递到CP0,由CP0统一处理。
    \item \textbf{寄存器访问控制}:通过MTC0/MFC0指令访问CP0寄存器,使用写掩码(WMASK)控制可写位域。
    \item \textbf{异常优先级}:在WB级进行异常仲裁,确定优先级后统一传递给CP0。
    \item \textbf{中断检测}:CP0内部检测定时器中断条件,输出中断信号。
\end{enumerate}

\subsection{CP0寄存器实现}

\subsubsection{寄存器列表}

本实现包含以下CP0寄存器:

\begin{table}[h]
\centering
\caption{CP0寄存器列表}
\begin{tabular}{llll}
\toprule
寄存器号 & 选择域 & 名称 & 功能描述 \\
\midrule
12 & 0 & STATUS & 系统状态寄存器 \\
13 & 0 & CAUSE & 异常原因寄存器 \\
14 & 0 & EPC & 异常程序计数器 \\
8 & 0 & BADVADDR & 错误虚拟地址寄存器 \\
9 & 0 & COUNT & 定时器计数寄存器 \\
11 & 0 & COMPARE & 定时器比较寄存器 \\
\bottomrule
\end{tabular}
\end{table}

\subsubsection{STATUS寄存器(寄存器12)}

STATUS寄存器用于控制系统状态和中断使能。

\begin{lstlisting}[caption=STATUS寄存器位域定义]
// STATUS寄存器位域
wire status_ie;        // bit 0: 全局中断使能 (IE)
wire status_exl;       // bit 1: 异常级别 (EXL)
wire [7:0] status_im;  // bit 15:8: 中断屏蔽位 (IM)

// STATUS寄存器写掩码(支持IE、EXL、IM位)
wire [31:0] STATUS_WMASK;
assign STATUS_WMASK = 32'h0000_8103; // bit 0(IE), bit 1(EXL), bit 15:8(IM)
\end{lstlisting}

\textbf{关键位域说明:}
\begin{itemize}
    \item \texttt{IE (bit 0)}:全局中断使能位。当IE=0时,所有中断被屏蔽。
    \item \texttt{EXL (bit 1)}:异常级别位。当EXL=1时,CPU处于异常处理模式,新的中断和异常被屏蔽。
    \item \texttt{IM[7:0] (bit 15:8)}:中断屏蔽位。每一位对应一个中断源,IM[7]对应定时器中断。
\end{itemize}

\textbf{写入控制:}
\begin{lstlisting}[caption=STATUS寄存器写入逻辑]
if (status_wen) begin
    status <= (status & ~STATUS_WMASK) | (wdata & STATUS_WMASK);
end
\end{lstlisting}

\subsubsection{CAUSE寄存器(寄存器13)}

CAUSE寄存器记录异常原因和中断状态。

\begin{lstlisting}[caption=CAUSE寄存器位域定义]
// CAUSE寄存器位域
wire cause_bd;         // bit 31: 延迟槽标志 (BD)
wire cause_ti;         // bit 30: 定时器中断标志 (TI)
wire [7:0] cause_ip;   // bit 15:8: 中断挂起位 (IP)
wire [4:0] cause_excode; // bit 6:2: 异常编码 (ExcCode)

// CAUSE寄存器写掩码(支持IP[1:0]位)
wire [31:0] CAUSE_WMASK;
assign CAUSE_WMASK = 32'h0000_0300;  // bit 9:8(IP[1:0])
\end{lstlisting}

\textbf{关键位域说明:}
\begin{itemize}
    \item \texttt{BD (bit 31)}:延迟槽标志。当异常发生在分支指令的延迟槽中时,BD=1。
    \item \texttt{TI (bit 30)}:定时器中断标志。当COUNT == COMPARE时,TI=1。
    \item \texttt{IP[7:0] (bit 15:8)}:中断挂起位。IP[7]对应定时器中断,由硬件自动更新。
    \item \texttt{ExcCode (bit 6:2)}:异常编码,标识异常类型。
\end{itemize}

\textbf{异常编码表:}
\begin{table}[h]
\centering
\caption{异常编码表}
\begin{tabular}{ll}
\toprule
异常编码 & 异常类型 \\
\midrule
0 & 中断 (Interrupt) \\
4 & 地址错误-加载 (AdEL) \\
5 & 地址错误-存储 (AdES) \\
8 & 系统调用 (SYSCALL) \\
9 & 断点 (BREAK) \\
12 & 算术溢出 (OV) \\
\bottomrule
\end{tabular}
\end{table}

\subsubsection{EPC寄存器(寄存器14)}

EPC寄存器保存发生异常时的程序计数器值。

\begin{lstlisting}[caption=EPC寄存器写入逻辑]
// 统一异常处理:所有异常都通过ex_valid_i传递
if (ex_valid_i && wb_valid) begin
    status[1] <= 1'b1;                     // EXL
    cause[31] <= ex_bd_i;                  // BD
    cause[6:2] <= ex_code_i;               // ExcCode
    epc <= ex_bd_i ? ex_pc_i : ex_pc_i;   // 写入分支PC或出错PC
    if (badvaddr_valid_i) begin
        badvaddr <= badvaddr_i;
    end
end
\end{lstlisting}

\textbf{说明:}
\begin{itemize}
    \item 当异常发生时,EPC保存发生异常的指令地址。
    \item 如果异常发生在延迟槽中(BD=1),EPC保存分支指令的地址。
    \item ERET指令执行时,CPU跳转到EPC保存的地址继续执行。
\end{itemize}

\subsubsection{BADVADDR寄存器(寄存器8)}

BADVADDR寄存器保存导致地址错误的虚拟地址。

\begin{lstlisting}[caption=BADVADDR寄存器写入逻辑]
if (badvaddr_valid_i) begin
    badvaddr <= badvaddr_i;
end
\end{lstlisting}

\textbf{说明:}
\begin{itemize}
    \item 仅在地址错误异常(AdEL/AdES)时写入。
    \item 由MEM级检测地址对齐错误,通过异常总线传递到CP0。
\end{itemize}

\subsubsection{COUNT寄存器(寄存器9)}

COUNT寄存器是定时器计数器,用于定时器中断。

\begin{lstlisting}[caption=COUNT寄存器实现]
// 定时器时钟分频(每两个时钟周期翻转一次,降低计数频率)
reg time_tick;
always @(posedge clk) begin
    if (!resetn) begin
        time_tick <= 1'b0;
    end else begin
        time_tick <= ~time_tick;
    end
end

// COUNT寄存器:可写,或每两个时钟周期自增
always @(posedge clk) begin
    if (!resetn) begin
        count <= 32'd0;
    end else begin
        if (count_wen) begin
            count <= wdata;
        end else if (time_tick) begin
            count <= count + 1'b1;
        end
    end
end
\end{lstlisting}

\textbf{说明:}
\begin{itemize}
    \item COUNT寄存器可通过MTC0指令写入。
    \item 正常情况下,每两个时钟周期自增1(通过\texttt{time\_tick}分频)。
    \item 当COUNT == COMPARE时,触发定时器中断。
\end{itemize}

\subsubsection{COMPARE寄存器(寄存器11)}

COMPARE寄存器是定时器比较值,用于定时器中断。

\begin{lstlisting}[caption=COMPARE寄存器实现]
// COMPARE寄存器:可写,写入时清除定时器中断
always @(posedge clk) begin
    if (!resetn) begin
        compare <= 32'd0;
    end else begin
        if (compare_wen) begin
            compare <= wdata;
        end
    end
end

// 定时器中断标志(cause_ti_reg)
always @(posedge clk) begin
    if (!resetn) begin
        cause_ti_reg <= 1'b0;
    end else begin
        if (compare_wen) begin
            cause_ti_reg <= 1'b0;  // 写入COMPARE时清除
        end else if (count_eq_compare) begin
            cause_ti_reg <= 1'b1;   // COUNT == COMPARE时置位
        end
    end
end
\end{lstlisting}

\textbf{说明:}
\begin{itemize}
    \item COMPARE寄存器可通过MTC0指令写入。
    \item 写入COMPARE寄存器会清除定时器中断标志(TI位)。
    \item 当COUNT == COMPARE时,置位TI标志,触发中断。
\end{itemize}

\subsection{异常处理流程}

\subsubsection{异常检测与仲裁}

异常检测分布在流水线的不同阶段:

\begin{enumerate}
    \item \textbf{ID级}:检测SYSCALL、BREAK指令。
    \item \textbf{MEM级}:检测地址对齐错误(AdEL/AdES)。
    \item \textbf{WB级}:统一仲裁所有异常,确定优先级。
    \item \textbf{CP0}:检测定时器中断。
\end{enumerate}

\subsubsection{WB级异常仲裁}

WB级负责统一仲裁所有异常,确定优先级后传递给CP0:

\begin{lstlisting}[caption=WB级异常仲裁逻辑]
// 异常仲裁逻辑(优先级:中断 > 地址错 > BREAK > SYSCALL)
// 非中断异常(地址错、BREAK、SYSCALL)
assign wb_ex_valid_no_int  = (mem_ex_adel_wb | mem_ex_ades_wb | brk_wb | syscall) ? WB_valid : 1'b0;
assign wb_ex_code_no_int   = mem_ex_adel_wb ? 5'd4 :  // AdEL
                             mem_ex_ades_wb ? 5'd5 :  // AdES
                             brk_wb ? 5'd9 :          // BREAK
                             syscall ? 5'd8 : 5'd0;   // SYSCALL

// 最终异常有效信号(中断优先级最高)
assign wb_ex_valid = (cp0_int && WB_valid) | wb_ex_valid_no_int;
assign wb_ex_code  = cp0_int ? 5'd0 : wb_ex_code_no_int;  // 中断异常码为0
assign wb_ex_bd    = 1'b0;       // 延迟槽后续接入
assign wb_ex_pc    = pc;         // 异常PC(地址错与syscall均取当前pc)
\end{lstlisting}

\textbf{异常优先级:}
\begin{enumerate}
    \item 定时器中断(最高优先级)
    \item 地址错误(AdEL/AdES)
    \item BREAK异常
    \item SYSCALL异常
\end{enumerate}

\subsubsection{异常处理流程}

当异常发生时,CP0执行以下操作:

\begin{enumerate}
    \item \textbf{设置EXL位}:STATUS[1] = 1,进入异常处理模式。
    \item \textbf{保存EPC}:将发生异常的PC保存到EPC寄存器。
    \item \textbf{设置CAUSE}:写入异常编码(ExcCode)和延迟槽标志(BD)。
    \item \textbf{保存BADVADDR}:如果是地址错误,保存错误地址。
    \item \textbf{跳转异常入口}:CPU跳转到异常入口地址(0x0)。
\end{enumerate}

\begin{lstlisting}[caption=CP0异常处理逻辑]
// 统一异常处理:所有异常都通过ex_valid_i传递
if (ex_valid_i && wb_valid) begin
    status[1] <= 1'b1;                     // EXL
    cause[31] <= ex_bd_i;                  // BD
    cause[6:2] <= ex_code_i;               // ExcCode
    epc <= ex_bd_i ? ex_pc_i : ex_pc_i;   // 写入分支PC或出错PC
    if (badvaddr_valid_i) begin
        badvaddr <= badvaddr_i;
    end
end
\end{lstlisting}

\subsubsection{ERET指令处理}

ERET指令用于从异常处理返回:

\begin{lstlisting}[caption=ERET指令处理]
// ERET指令:清除EXL位
if (eret && wb_valid) begin
    status[1] <= 1'b0;   // 清EXL
end

// 异常/返回对外信号
assign exc_pc = eret ? epc : `EXC_ENTER_ADDR;
\end{lstlisting}

\textbf{说明:}
\begin{itemize}
    \item ERET执行时,清除STATUS[1](EXL位),退出异常处理模式。
    \item CPU跳转到EPC保存的地址继续执行。
\end{itemize}

\subsection{中断处理}

\subsubsection{定时器中断检测}

定时器中断由CP0内部检测:

\begin{lstlisting}[caption=定时器中断检测逻辑]
// COUNT == COMPARE检测
assign count_eq_compare = (count == compare);

// 定时器中断标志(cause_ti_reg)
always @(posedge clk) begin
    if (!resetn) begin
        cause_ti_reg <= 1'b0;
    end else begin
        if (compare_wen) begin
            cause_ti_reg <= 1'b0;  // 写入COMPARE时清除
        end else if (count_eq_compare) begin
            cause_ti_reg <= 1'b1;   // COUNT == COMPARE时置位
        end
    end
end

// CAUSE寄存器位域更新
always @(posedge clk) begin
    // cause[30]: TI位(定时器中断标志)
    if (!ex_valid_i || !wb_valid) begin
        cause[30] <= cause_ti_reg;
    end
    
    // cause[15:8]: IP位(中断挂起位)
    // IP[7] = TI(定时器中断)
    if (!ex_valid_i || !wb_valid) begin
        cause[15:8] <= {cause_ti_reg, 5'd0, cause[9:8]};
    end
end
\end{lstlisting}

\subsubsection{中断使能条件}

中断需要满足以下条件才能触发:

\begin{lstlisting}[caption=中断检测逻辑]
// 中断检测逻辑
// 中断条件:有中断挂起 && 对应中断使能 && 全局中断使能 && 不在异常级别
assign c0_int = |(cause_ip[7:0] & status_im[7:0]) & status_ie & !status_exl;
\end{lstlisting}

\textbf{中断使能条件:}
\begin{enumerate}
    \item 有中断挂起(IP位为1)
    \item 对应中断屏蔽位使能(IM位为1)
    \item 全局中断使能(IE=1)
    \item 不在异常级别(EXL=0)
\end{enumerate}

\subsection{总线排布}

\subsubsection{MEM->WB总线}

MEM级通过总线将异常信息传递给WB级:

\begin{lstlisting}[caption=MEM->WB总线定义]
`define MEM_WB_BUS_WIDTH    153

// 扩展MEM->WB总线,新增:mem_ex_adel, mem_ex_ades, badvaddr(dm_addr)
assign MEM_WB_bus = {rf_wen,rf_wdest,                   // WB需要使用的信号
                     mem_result,                        // 最终要写回寄存器的数据
                     lo_result,                         // 乘法低32位结果
                     hi_write,lo_write,                 // HI/LO写使能
                     mfhi,mflo,                         // WB需要使用的信号
                     mtc0,mfc0,cp0r_addr,syscall,brk,eret,  // WB需要使用的信号
                     mem_ex_adel, mem_ex_ades,          // 地址异常标志(新增)
                     dm_addr,                           // BADVADDR(新增)
                     pc};                               // PC值
\end{lstlisting}

\textbf{总线位域说明:}
\begin{itemize}
    \item \texttt{mem\_ex\_adel}:Load地址错误标志
    \item \texttt{mem\_ex\_ades}:Store地址错误标志
    \item \texttt{dm\_addr}:访存地址(用于BADVADDR)
    \item \texttt{syscall, brk, eret}:异常指令标识
\end{itemize}

\subsubsection{WB->CP0异常总线}

WB级通过异常总线将异常信息传递给CP0:

\begin{lstlisting}[caption=WB->CP0异常总线]
// 统一异常总线(传递给CP0)
assign wb_ex_valid = (cp0_int && WB_valid) | wb_ex_valid_no_int;
assign wb_ex_code  = cp0_int ? 5'd0 : wb_ex_code_no_int;
assign wb_ex_bd    = 1'b0;
assign wb_ex_pc    = pc;
assign wb_badvaddr_valid = mem_ex_adel_wb | mem_ex_ades_wb;
assign wb_badvaddr = mem_badvaddr_wb;
\end{lstlisting}

\subsection{各异常类型的实现}

\subsubsection{SYSCALL异常}

\textbf{检测位置:}ID级(指令译码)

\begin{lstlisting}[caption=SYSCALL指令检测]
// decode.v
assign inst_SYSCALL = (op == 6'b000000) & (funct == 6'b001100);
\end{lstlisting}

\textbf{处理流程:}
\begin{enumerate}
    \item ID级检测到SYSCALL指令,将\texttt{syscall}信号传递到WB级。
    \item WB级仲裁,设置异常码为8。
    \item CP0保存EPC(SYSCALL指令地址),设置EXL=1,跳转到异常入口。
    \item 异常处理程序读取CAUSE,判断为SYSCALL,执行相应处理。
    \item 处理完成后,EPC += 4,ERET返回。
\end{enumerate}

\subsubsection{BREAK异常}

\textbf{检测位置:}ID级(指令译码)

\begin{lstlisting}[caption=BREAK指令检测]
// decode.v
assign inst_BREAK = (op == 6'b000000) & (funct == 6'b001101);
\end{lstlisting}

\textbf{处理流程:}
\begin{enumerate}
    \item ID级检测到BREAK指令,将\texttt{brk}信号传递到WB级。
    \item WB级仲裁,设置异常码为9。
    \item CP0保存EPC(BREAK指令地址),设置EXL=1,跳转到异常入口。
    \item 异常处理程序读取CAUSE,判断为BREAK,执行相应处理。
    \item 处理完成后,EPC += 4,ERET返回。
\end{enumerate}

\subsubsection{地址错误异常(AdEL/AdES)}

\textbf{检测位置:}MEM级(访存阶段)

\begin{lstlisting}[caption=地址对齐错误检测]
// mem.v
wire mem_ex_adel;  // load 地址错
wire mem_ex_ades;  // store 地址错
assign mem_ex_adel = MEM_valid && inst_load  && ls_word && (dm_addr[1:0]!=2'b00);
assign mem_ex_ades = MEM_valid && inst_store && ls_word && (dm_addr[1:0]!=2'b00);
\end{lstlisting}

\textbf{处理流程:}
\begin{enumerate}
    \item MEM级检测到地址不对齐(低2位不为00),设置\texttt{mem\_ex\_adel}或\texttt{mem\_ex\_ades}。
    \item 将错误地址(\texttt{dm\_addr})和异常标志传递到WB级。
    \item WB级仲裁,设置异常码为4(AdEL)或5(AdES)。
    \item CP0保存EPC(出错指令地址),保存BADVADDR(错误地址),设置EXL=1,跳转到异常入口。
    \item 异常处理程序读取CAUSE和BADVADDR,执行相应处理。
    \item 处理完成后,EPC += 4,ERET返回。
\end{enumerate}

\subsubsection{定时器中断}

\textbf{检测位置:}CP0内部

\begin{lstlisting}[caption=定时器中断检测]
// cp0.v
assign count_eq_compare = (count == compare);
assign c0_int = |(cause_ip[7:0] & status_im[7:0]) & status_ie & !status_exl;
\end{lstlisting}

\textbf{处理流程:}
\begin{enumerate}
    \item CP0检测到COUNT == COMPARE,置位TI标志。
    \item 检查中断使能条件(IE=1, IM[7]=1, EXL=0)。
    \item 如果条件满足,输出\texttt{c0\_int}信号。
    \item WB级仲裁,设置异常码为0(中断)。
    \item CP0保存EPC(被中断指令地址),设置EXL=1,跳转到异常入口。
    \item 异常处理程序读取CAUSE,检查TI位,执行定时器中断处理。
    \item 处理完成后,写入COMPARE清除TI位,ERET返回。
\end{enumerate}

\subsection{CP0寄存器访问}

\subsubsection{MTC0指令(写入CP0寄存器)}

MTC0指令用于写入CP0寄存器:

\begin{lstlisting}[caption=MTC0指令处理]
// 写允许信号
wire mtc0_wr;  // MTC0写使能(排除异常时写入)
assign mtc0_wr = mtc0 && wb_valid && !ex_valid_i; // 异常时不写入

assign status_wen   = mtc0_wr && sel_status;
assign cause_wen    = mtc0_wr && sel_cause;
assign epc_wen      = mtc0_wr && sel_epc;
assign count_wen    = mtc0_wr && sel_count;
assign compare_wen  = mtc0_wr && sel_compare;
assign badvaddr_wen = mtc0_wr && sel_badvaddr;

// 写入逻辑(使用写掩码)
if (status_wen) begin
    status <= (status & ~STATUS_WMASK) | (wdata & STATUS_WMASK);
end
\end{lstlisting}

\textbf{说明:}
\begin{itemize}
    \item 异常发生时,禁止写入CP0寄存器(\texttt{!ex\_valid\_i})。
    \item 使用写掩码(WMASK)控制可写位域。
    \item STATUS和CAUSE寄存器只有部分位可写。
\end{itemize}

\subsubsection{MFC0指令(读取CP0寄存器)}

MFC0指令用于读取CP0寄存器:

\begin{lstlisting}[caption=MFC0指令处理]
// MFC0读
assign cp0r_rdata = sel_status  ? status   :
                    sel_cause   ? cause    :
                    sel_epc     ? epc      :
                    sel_count   ? count    :
                    sel_compare ? compare  :
                    sel_badvaddr? badvaddr : 32'd0;
\end{lstlisting}

\textbf{说明:}
\begin{itemize}
    \item 根据CP0寄存器地址(\texttt{cp0r\_addr})选择对应的寄存器。
    \item 读数据通过\texttt{cp0r\_rdata}输出,写回到通用寄存器。
\end{itemize}

\subsection{流水线控制信号}

\subsubsection{cancel信号}

\texttt{cancel}信号用于取消流水线中已取出的指令:

\begin{lstlisting}[caption=cancel信号生成]
assign cancel = (ex_valid_i | eret | c0_int) && wb_over;
\end{lstlisting}

\textbf{说明:}
\begin{itemize}
    \item 当异常或ERET发生时,需要取消流水线中已取出的指令。
    \item \texttt{cancel}信号在WB级完成时(\texttt{wb\_over})发出。
\end{itemize}

\subsubsection{exc\_valid和exc\_pc信号}

\texttt{exc\_valid}和\texttt{exc\_pc}信号用于控制异常跳转:

\begin{lstlisting}[caption=异常跳转信号生成]
assign exc_valid = (ex_valid_i | eret | c0_int) && wb_valid;
assign exc_pc    = eret ? epc : `EXC_ENTER_ADDR;
\end{lstlisting}

\textbf{说明:}
\begin{itemize}
    \item \texttt{exc\_valid}:异常有效信号,控制是否跳转。
    \item \texttt{exc\_pc}:跳转目标地址,ERET时返回EPC,异常时跳转到异常入口(0x0)。
\end{itemize}

\subsection{软件异常处理程序}

\textbf{注意:}本章节预留,用于描述异常处理程序的编写方法、测试程序的设计思路等。具体内容待补充。

\subsubsection{异常处理程序框架}

(待补充)

\subsubsection{测试程序设计}

(待补充)

\section{实验总结}

\begin{enumerate}
    \item \textbf{CP0模块设计}:成功实现了CP0模块,包含6个CP0寄存器(STATUS、CAUSE、EPC、BADVADDR、COUNT、COMPARE),每个寄存器都有明确的位域定义和访问控制机制。
    
    \item \textbf{异常处理机制}:实现了统一的异常处理流程,通过WB级异常仲裁确定异常优先级,所有异常都通过统一的异常总线传递给CP0,由CP0统一处理。异常发生时,CP0自动保存EPC、设置EXL位、写入CAUSE寄存器等。
    
    \item \textbf{异常类型实现}:成功实现了4种异常类型:
    \begin{itemize}
        \item SYSCALL异常:在ID级检测,异常码为8
        \item BREAK异常:在ID级检测,异常码为9
        \item 地址错误异常(AdEL/AdES):在MEM级检测,异常码为4/5
        \item 定时器中断:在CP0内部检测,异常码为0
    \end{itemize}
    
    \item \textbf{中断处理}:实现了定时器中断机制,包括COUNT/COMPARE寄存器的实现、定时器中断检测逻辑、中断使能条件判断等。定时器中断可以在满足条件时异步触发,优先级最高。
    
    \item \textbf{总线设计}:设计了MEM->WB总线和WB->CP0异常总线,实现了异常信息在流水线中的正确传递。总线宽度为153位,包含了所有必要的异常信号。
    
    \item \textbf{寄存器访问}:实现了MTC0/MFC0指令,支持CP0寄存器的读写访问。使用写掩码(WMASK)控制可写位域,确保寄存器的安全性。
    
    \item \textbf{流水线控制}:实现了cancel、exc\_valid、exc\_pc等流水线控制信号,确保异常发生时流水线能够正确响应,取消已取出的指令,跳转到异常入口或ERET返回地址。
    
    \item \textbf{设计规范}:整个实现遵循MIPS架构规范,提供了完整的异常处理和中断支持,为系统软件提供了可靠的异常处理机制。
\end{enumerate}

\label{LastPage}

\end{document}
