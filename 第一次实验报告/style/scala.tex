% "define" Scala
\usepackage[T1]{fontenc}  
\usepackage[scaled=0.82]{beramono}  
\usepackage{microtype} 

\sbox0{\small\ttfamily A}
\edef\mybasewidth{\the\wd0 }

\lstdefinelanguage{scala}{
  morekeywords={abstract,case,catch,class,def,%
    do,else,extends,false,final,finally,%
    for,if,implicit,import,match,mixin,%
    new,null,object,override,package,%
    private,protected,requires,return,sealed,%
    super,this,throw,trait,true,try,%
    type,val,var,while,with,yield},
  sensitive=true,
  morecomment=[l]{//},
  morecomment=[n]{/*}{*/},
  morestring=[b]",
  morestring=[b]',
  morestring=[b]"""
}

\usepackage{color}
\definecolor{dkgreen}{rgb}{0,0.6,0}
\definecolor{gray}{rgb}{0.5,0.5,0.5}
\definecolor{mauve}{rgb}{0.58,0,0.82}

% Default settings for code listings
\lstset{frame=tb,
  language=scala,
  aboveskip=3mm,
  belowskip=3mm,
  showstringspaces=false,
  columns=fixed, % basewidth=\mybasewidth,
  basicstyle={\small\ttfamily},
  numbers=none,
  numberstyle=\footnotesize\color{gray},
  % identifierstyle=\color{red},
  keywordstyle=\color{blue},
  commentstyle=\color{dkgreen},
  stringstyle=\color{mauve},
  frame=single,
  breaklines=true,
  breakatwhitespace=true,
  procnamekeys={def, val, var, class, trait, object, extends},
  procnamestyle=\ttfamily\color{red},
  tabsize=2
}

\lstnewenvironment{scala}[1][]
{\lstset{language=scala,#1}}
{}
\lstnewenvironment{cpp}[1][]
{\lstset{language=C++,#1}}
{}
\lstnewenvironment{bash}[1][]
{\lstset{language=bash,#1}}
{}
\lstnewenvironment{verilog}[1][]
{\lstset{language=verilog,#1}}
{}



\graphicspath{ {images/} }
\usepackage{ctex}
\usepackage{graphicx}
\usepackage{color,framed}%文本框
\usepackage{listings}
\usepackage{caption}

\usepackage{hyperref}
\hypersetup{hidelinks,
	colorlinks=true,
	allcolors=black,
	pdfstartview=Fit,
	breaklinks=true}


\lstdefinestyle{mystyle}{
  keywordstyle=\color{orange},  % 设置关键词颜色为橙色
 morekeywords={set, nmap, msfconsole,exploit},           % 仅将 set 设置为橙色
}

\usepackage{amssymb}
\usepackage{enumerate}
\usepackage{xcolor}
\usepackage{bm} 
\usepackage{lastpage}%获得总页数
\usepackage{fancyhdr}
\usepackage{tabularx}  
\usepackage{geometry}
\usepackage{minted}
\usepackage{graphics}
\usepackage{subfigure}
\usepackage{float}
\usepackage{pdfpages}
\usepackage{pgfplots}
\pgfplotsset{width=10cm,compat=1.9}
\usepackage{multirow}
\usepackage{footnote}
\usepackage{booktabs}
\usepackage{listings}

%-----------------------伪代码------------------
\usepackage{algorithm}  
\usepackage{algorithmicx}  
\usepackage{algpseudocode}  
\floatname{algorithm}{Algorithm}  
\renewcommand{\algorithmicrequire}{\textbf{Input:}}  
\renewcommand{\algorithmicensure}{\textbf{Output:}} 
\usepackage{lipsum}  
\makeatletter
\newenvironment{breakablealgorithm}
  {% \begin{breakablealgorithm}
  \begin{center}
     \refstepcounter{algorithm}% New algorithm
     \hrule height.8pt depth0pt \kern2pt% \@fs@pre for \@fs@ruled
     \renewcommand{\caption}[2][\relax]{% Make a new \caption
      {\raggedright\textbf{\ALG@name~\thealgorithm} ##2\par}%
      \ifx\relax##1\relax % #1 is \relax
         \addcontentsline{loa}{algorithm}{\protect\numberline{\thealgorithm}##2}%
      \else % #1 is not \relax
         \addcontentsline{loa}{algorithm}{\protect\numberline{\thealgorithm}##1}%
      \fi
      \kern2pt\hrule\kern2pt
     }
  }{% \end{breakablealgorithm}
     \kern2pt\hrule\relax% \@fs@post for \@fs@ruled
  \end{center}
  }
\makeatother
%------------------------代码-------------------
\RequirePackage{listings}
\RequirePackage{xcolor}
\definecolor{dkgreen}{rgb}{0,0.6,0}
\definecolor{gray}{rgb}{0.5,0.5,0.5}
\definecolor{mauve}{rgb}{0.58,0,0.82}
\lstset{
	numbers=left,  
	frame=tb,
	aboveskip=3mm,
	belowskip=3mm,
	showstringspaces=false,
	columns=flexible,
	framerule=1pt,
	rulecolor=\color{gray!35},
	backgroundcolor=\color{gray!5},
	basicstyle={\ttfamily},
	numberstyle=\tiny\color{gray},
	keywordstyle=\color{blue},
	commentstyle=\color{dkgreen},
	stringstyle=\color{mauve},
	breaklines=true,
	breakatwhitespace=true,
	tabsize=3,
}


%-------------------------页面边距--------------
\geometry{a4paper,left=2.3cm,right=2.3cm,top=2.7cm,bottom=2.7cm}
%-------------------------页眉页脚--------------
\usepackage{fancyhdr}
\pagestyle{fancy}
\lhead{\kaishu }
\chead{}
\rhead{{\CJKfontspec{simkai.ttf} }}
\lfoot{}
\cfoot{\thepage}
\rfoot{}
\renewcommand{\headrulewidth}{0pt}  
\renewcommand{\footrulewidth}{0pt}
\newcommand{\HRule}{\rule{\linewidth}{0.5mm}}
\newcommand{\HRulegrossa}{\rule{\linewidth}{1.2mm}}
\setlength{\textfloatsep}{10mm}
%--------------------文档内容--------------------
\renewcommand{\contentsname}{目\ 录}
\renewcommand{\appendixname}{附录}
\renewcommand{\appendixpagename}{附录}
\renewcommand{\refname}{参考文献} 
\renewcommand{\figurename}{图}
\renewcommand{\tablename}{表}
\renewcommand{\today}{\number\year 年 \number\month 月 \number\day 日}

\renewcommand {\thefigure}{\thesection{}.\arabic{figure}}%图片按章标号
\renewcommand{\figurename}{图}
\renewcommand{\contentsname}{目录}  
\cfoot{\thepage\ of \pageref{LastPage}}%当前页 of 总页数

\renewcommand{\abstractname}{\textbf{\Large 摘要}} % 调整摘要标题的字体大小