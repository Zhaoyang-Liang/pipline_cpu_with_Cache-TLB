\documentclass{article}
\usepackage{ctex}
\usepackage{geometry}
\usepackage{graphicx}
\usepackage{booktabs}
\usepackage{float}
\usepackage{xcolor}
\usepackage{listings}

\title{在五级流水CPU上实现AXI总线}
\author{梁朝阳}

\lstdefinelanguage{Verilog}{
  keywords={module, endmodule, input, output, wire, reg, always, begin, end, assign,
            if, else, case, endcase, parameter, function},
  keywordstyle=\color{blue}\bfseries,
  basicstyle=\ttfamily\small,
  comment=[l]{//},
  commentstyle=\color{green!50!black},
  morecomment=[s]{/*}{*/},
  stringstyle=\color{orange},
  showstringspaces=false
}

\lstset{
  frame=single,            % 添加黑色边框
  rulecolor=\color{black}, % 设置边框为黑色
  framerule=0.8pt          % 边框线粗细(可选)
}


\begin{document}
\section{实验要求}
参考《CPU设计实战》的第8章,自己动手实现总线接口相关功能。要求:
1、根据自己情况,类SRAM总线和AXI总线完成一项支持即可,建议选择AXI总线(需要查阅AXI总线文档)。
2、虽然说不建议使用当前五级流水的代码了,但是我学习后想保持一个实验的连续性,而且我想把这个CPU实现完全,因为还是采用了这个CPU的代码。

\section{AXI原理}
\section{总览}

AXI 总线是一种突发总线,突发传输。一直连续的传输,比如说突发8次传输,就是指传输数据连续的传输。(突发传输的地址是连续的,数据是连续的)

\begin{enumerate}
    \item 读:地址控制(主机(Master)发送读地址)-> 从机(Slave)发送数据。
    \item 写:地址控制(主机发送读地址)-> 主机发送数据 -> 接受完后response。
\end{enumerate}

\section{AXI的五个通道的信号}

\subsection{全局信号}
\begin{table}[H]
    \centering
    \caption{全局信号 | Global Signals}
    \label{tab:global_signals}
    \begin{tabular}{cccc}
        \toprule
        \textbf{信号} & \textbf{来源} & \textbf{描述} \\
        \midrule
        ACLK & 时钟源 & 全局时钟信号,所有信号在时钟上升沿采样 \\
        ARESETn & 复位源 & 全局复位信号,低电平有效\\
        \bottomrule
    \end{tabular}
\end{table}

\subsection{写地址}
\begin{table}[H]
    \centering
    \caption{写地址通道信号|Write address channel signals}
    \label{tab:write_address_channel_signals}
    \begin{tabular}{cccc}
        \toprule
        \textbf{通道}&  \textbf{来源} & \textbf{宽度} & \textbf{描述} \\
        \midrule
        AWID[3:0] & Master & 4 & 写地址ID \\
        \textbf{\textcolor{red}{AWADDR[31:0]}} & Master & 32 & 写地址 \\
        \textbf{\textcolor{red}{AWLEN[3:0]}} & Master & 4 & 给出了突发传输几次 \\
        \textbf{\textcolor{red}{AWSIZE[2:0]}} & Master & 3 & 突发传输的数据宽度 。例如有32个字节,那么size是4 。\\
        AWBURST[1:0] & Master & 2 & 突发传输类型 $\dagger $\\
        AWLOCK[1:0] & -- & 2 & AXI不支持锁 \\
        AWCACHE[3:0] & Master & 4 & cache类型 \\
        AWPROT[2:0] & Master & 3 & Protection type. (不是端口,而是保护) \\
        AWVALID & Master & 1 & 写地址有效 \\    
        AWREADY & Slave & 1 & 写地址准备好 $\ddagger$ \\
        \bottomrule
    \end{tabular}
\end{table}
$\dagger$ 固定自增扫描打包,00: fixed, 01: increment, 10: wrapping. 

$\ddagger$ AWVALID \& AWREADY为1 时可以数据有效

\subsection{写数据}

\begin{table}[H]
    \centering
    \caption{写地址通道信号|Write address channel signals}
    \label{tab:write_address_channel_signals}
    \begin{tabular}{cccc}
        \toprule
        \textbf{通道}&  \textbf{来源} & \textbf{宽度} & \textbf{描述} \\
        \midrule
        WID[3:0] & Master & 4 & 写数据ID \\
        \textbf{\textcolor{red}{WDATA[31:0]}} & Master & 32 & 写数据 \\
        \textbf{\textcolor{red}{WSTRB[3:0]}} & Master & 4 & (掩码/闪光)写数据有效字节 $\dagger$ \\
        \textbf{\textcolor{red}{WLAST}} & Master & 1 & 写数据最后一个 \\
        WVALID & Master & 1 & 写数据有效 \\
        WREADY & Slave & 1 & 写数据准备好 \\
        \bottomrule
    \end{tabular}
\end{table}
$\dagger$ 例如数据:32'H1234\_5678 和掩码:4'BO111 代表:12 无效,其余有效。 WSTRB[n] correspondsto WDATA[(8 × n) + 7:(8 × n)].

\subsection{写响应}
\begin{table}[H]
    \centering
    \caption{写响应通道信号|Write response channel signals}
    \label{tab:write_response_channel_signals}
    \begin{tabular}{cccc}
        \toprule
        \textbf{通道}&  \textbf{来源} & \textbf{宽度} & \textbf{描述} \\
        \midrule
        BID[3:0] & Slave & 4 & 写响应ID \\
        \textbf{\textcolor{red}{BRESP[1:0]}} & Slave & 2 & 写响应状态 $\dagger$ \\
        BVALID & Slave & 1 & 写响应有效 \\
        BREADY & Master & 1 & 写响应准备好 \\
        \bottomrule
    \end{tabular}
\end{table}
$\dagger$ 写响应状态:00: OKAY, 01: EXOKAY, 10: SLVERR, 11: DECERR。

\subsection{读地址}
几乎完全和\ref{tab:write_address_channel_signals}一样。

\subsection{读数据}

把RESP信号和写数据信号并在一起了,功能类似写信号。

\subsection{低功耗接口信号}

\begin{table}[H]
    \centering
    \caption{低功耗接口信号(Low Power Interface Signals)}
    \label{tab:low_power_interface_signals}
    \begin{tabular}{cccc}
        \toprule
        \textbf{信号名称}  & \textbf{来源}             & \textbf{描述} \\
        \midrule
        CSYSREQ        & 时钟控制器(Clock controller) & SYS low-pow Request\\
        CSYSACK        & 外设(Peripheral device)      & ACK \\
        CACTIVE        & 外设(Peripheral device)      & 时钟活跃指示。$\dagger$ \\
        \bottomrule
    \end{tabular}
\end{table}
$\dagger$ 1=需要时钟,0=不需要时钟。表示外设是否需要其时钟信号。

\section{代码实现历程}
主要包含:一个完整可用的 AXI4-Full Master 与 AXI4-Compatible RAM Slave 的 RTL 设计。Master 能够执行可配置的读写突发操作;Slave 则模拟一块 AXI4-Full 协议下的 RAM 存储器,能够正确响应 AXI 的五条通道(AW/W/B/AR/R)。

\begin{itemize}
    \item AXI Master 的接口定义、参数设计、功能说明
    \item 内部状态机设计与逻辑划分
    \item 关键模块实现(地址通道、数据通道、响应通道)
    \item 可用于测试的 AXI4-Full RAM Slave 的实现细节
    \item 设计中关键协议点的解释与讨论
\end{itemize}


\subsection{接口与参数定义}

AXI Master 模块的关键参数如下所示:

\begin{lstlisting}[language=Verilog]
module axi_full_master #(
    parameter C_M_AXI_ID_WIDTH     = 1,
    parameter C_M_AXI_ADDR_WIDTH   = 32,
    parameter C_M_AXI_DATA_WIDTH   = 32,
    parameter C_M_TARGET_SLAVE_BASE_ADDR = 32'h00000000
)(
    input  wire                        M_AXI_ACLK,
    input  wire                        M_AXI_ARESETN,
\end{lstlisting}

Master 的接口包括五条 AXI 通道及用户控制接口,用户接口用于控制:

\begin{itemize}
    \item 操作类型(读/写)
    \item 启动信号
    \item 访问起始地址
    \item 突发长度
    \item 写入数据通道
    \item 读出数据通道
\end{itemize}

这使得 Master 能够作为上层模块(CPU 模拟器或测试逻辑)的直接驱动对象。

\subsection{二进制位宽计算函数}

AXI 协议中的 AWSIZE/ARSIZE 字段需要根据数据宽度自动计算。
使用如下函数:

\begin{lstlisting}[language=Verilog]
function integer clogb2;
    input integer number;
    integer i;
begin
    clogb2 = 0;
    for(i = number-1; i > 0; i = i >> 1)
        clogb2 = clogb2 + 1;
end
endfunction
\end{lstlisting}

这对自动生成地址步长(数据字节宽度)至关重要。

\subsection{Master 内部寄存器与信号}

Master 的关键信号包括:

\begin{itemize}
    \item 地址锁存寄存器(读/写独立)
    \item 长度计数器
    \item 当前突发类型
    \item 写数据 beat 计数器
    \item 状态机状态寄存器
    \item 用户侧 busy/done/error 状态
\end{itemize}

示例:

\begin{lstlisting}[language=Verilog]
reg [C_M_AXI_ADDR_WIDTH-1:0] addr_reg;
reg [7:0]                    len_reg;
reg                          rw_reg;
reg [7:0]                    wbeat_cnt;
reg                          error_reg;
reg                          done_reg;
\end{lstlisting}

\subsection{协议状态机设计}

AXI Master 包含两套状态机:

\begin{enumerate}
    \item 写事务状态机:
    \begin{itemize}
        \item 写地址阶段(AW)
        \item 写数据阶段(W)
        \item 写响应阶段(B)
    \end{itemize}
    \item 读事务状态机:
    \begin{itemize}
        \item 读地址阶段(AR)
        \item 读数据阶段(R)
    \end{itemize}
\end{enumerate}

写状态机示例:

\begin{lstlisting}[language=Verilog]
localparam ST_IDLE = 3'd0,
           ST_AW   = 3'd1,
           ST_W    = 3'd2,
           ST_B    = 3'd3,
           ST_DONE = 3'd4;
\end{lstlisting}

Master 仅在用户发出 start 信号时进入事务,并在 B/R 通道结束后回到空闲状态。

写地址的控制例:

\begin{lstlisting}[language=Verilog]
always @(posedge M_AXI_ACLK) begin
    if (!M_AXI_ARESETN)
        M_AXI_AWVALID <= 1'b0;
    else if (state == ST_AW)
        M_AXI_AWVALID <= 1'b1;
    else if (M_AXI_AWVALID && M_AXI_AWREADY)
        M_AXI_AWVALID <= 1'b0;
end
\end{lstlisting}

读事务逻辑与其类似,但对应使用 AR/R 通道。

\subsection{Master 写数据逻辑示例}

写数据通道通过用户数据驱动:

\begin{lstlisting}[language=Verilog]
assign M_AXI_WDATA  = user_wdata;
assign M_AXI_WVALID = (state == ST_W) && user_wvalid;
assign M_AXI_WLAST  = (wbeat_cnt == len_reg - 1);
\end{lstlisting}

写响应处理:

\begin{lstlisting}[language=Verilog]
if (M_AXI_BVALID && M_AXI_BREADY) begin
    if (M_AXI_BRESP != 2'b00) error_reg <= 1'b1;
end
\end{lstlisting}

Master 将异常状态反馈到 user\_error。

\section{AXI RAM Slave 设计}

\subsection{设计目标}

Slave 必须做到:

\begin{itemize}
    \item 完全遵循 AXI4-Full 五通道协议
    \item 支持连续突发(INCR)、固定(FIXED)、回绕(WRAP)
    \item 支持字节写使能(WSTRB)
    \item 支持读写并发
    \item 可配置 RAM 深度
\end{itemize}

Slave 模块参数如下:

\begin{lstlisting}[language=Verilog]
parameter C_S_RAM_DEPTH = 256;
reg [C_S_AXI_DATA_WIDTH-1:0] ram [0:C_S_RAM_DEPTH-1];
\end{lstlisting}

\subsection{地址递增函数实现}

支持三种突发模式:

\begin{lstlisting}[language=Verilog]
function [C_S_AXI_ADDR_WIDTH-1:0] axi_next_addr;
    input [C_S_AXI_ADDR_WIDTH-1:0] addr;
    input [1:0] burst;
    input [2:0] size;
    input [7:0] len;
begin
    integer inc = (1 << size);
    integer bytes = inc * (len + 1);

    case (burst)
        2'b00: axi_next_addr = addr;            // FIXED
        2'b01: axi_next_addr = addr + inc;      // INCR
        2'b10: begin                            // WRAP
            reg [31:0] base = addr & ~(bytes-1);
            axi_next_addr = base |
               ((addr + inc) & (bytes-1));
        end
        default: axi_next_addr = addr + inc;
    endcase
end
endfunction
\end{lstlisting}

\subsection{写通道逻辑}

写地址握手:

\begin{lstlisting}[language=Verilog]
assign S_AXI_AWREADY = ~wr_active;
always @(posedge S_AXI_ACLK) begin
    if (aw_hs) begin
        wr_active <= 1;
        wr_addr_reg <= S_AXI_AWADDR;
        wr_len_reg  <= S_AXI_AWLEN;
    end
end
\end{lstlisting}

写数据:

\begin{lstlisting}[language=Verilog]
if (wr_active && w_hs) begin
    integer idx = wr_addr_reg[ADDR_LSB +: RAM_ADDR_WIDTH];
    for (i=0;i<C_S_AXI_DATA_WIDTH/8;i=i+1)
        if (S_AXI_WSTRB[i])
            ram[idx][8*i +: 8] <= S_AXI_WDATA[8*i +: 8];
end
\end{lstlisting}

\subsection{读通道逻辑}

读地址握手:

\begin{lstlisting}[language=Verilog]
assign S_AXI_ARREADY = ~rd_active;
if (ar_hs) begin
    rd_active <= 1;
    rd_addr_reg <= S_AXI_ARADDR;
    rd_len_reg  <= S_AXI_ARLEN;
end
\end{lstlisting}

读数据产生:

\begin{lstlisting}[language=Verilog]
integer idx_r = rd_addr_reg[ADDR_LSB +: RAM_ADDR_WIDTH];
rdata_reg <= ram[idx_r];
rvalid_reg <= 1;
rlast_reg <= (rd_cnt == rd_len_reg);
\end{lstlisting}

Slave 按协议连续输出数据直到最后一个 beat。



\end{document}



